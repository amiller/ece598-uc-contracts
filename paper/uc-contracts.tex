\documentclass[11pt]{article}
\usepackage[dvipsnames]{xcolor}
\usepackage{fancyhdr}
\usepackage{url}
\usepackage[most]{tcolorbox}
\usepackage{multirow}
\usepackage{subcaption}
\usepackage{setspace}
\usepackage{enumitem} 
\usepackage{amsthm}
\usepackage[ruled,vlined]{algorithm2e}
\usetikzlibrary{matrix, arrows.meta, calc, positioning}
\tikzset{myarrow/.style={-Latex, rounded corners},}

\definecolor{vert}{RGB}{0,181,0}
\definecolor{oran}{RGB}{223,74,0}
\definecolor{viol}{RGB}{134,0,175}
\definecolor{roug}{RGB}{215,15,0}
\definecolor{bb}{RGB}{0,0,0}
\definecolor{gg}{RGB}{220,220,220}

\newtcolorbox[auto counter]{bbox}[2][]{%
    colback=white,
    colframe=bb,
    %colbacktitle=white!90!roug,
	colbacktitle=white!40!gg,
    coltitle=black,
    fonttitle=\bfseries, 
    enhanced,
    attach boxed title to top left={yshift=-2mm, xshift=0.5cm},%
    #1,% For possible options
}
\topmargin=-5mm
\evensidemargin=0cm
\oddsidemargin=0cm
\textwidth=16cm
\textheight=22cm
\addtolength{\headheight}{1.6pt}
\newcommand{\cancel}[1]{}
\newcommand{\mymark}{$^*$}
\setlength\parindent{24pt}

\newcommand{\lastupdate}{August 2015}

%\lhead{\sc IACR Policy for Cryptology Schools}
%\rhead{\sc \lastupdate}

\title{\bf SAUCY}
\author{\mbox{Surya Bakshi, Andrew Miller}}

%\date{\lastupdate
% \footnote{The most recent version of this document
%    can be obtained from \protect\url{http://www.iacr.org/docs/}.\newline
%  Editors of this document: M. Abdalla, A. Boldyreva, C. Cachin, A. Kiayias, B. Warinschi (2014).}}

\newcommand{\msf}[1]{\ensuremath{{\mathsf {#1}}}}
\newcommand{\f}[1]{\ensuremath{\mathcal{#1}}}
\newcommand{\F}{\f{F}}
\newcommand{\C}{\mathcal{C}}
\newcommand{\con}[1]{\msf{Contract_{#1}}}
%\newcommand{\Fsync}[2]{\ensuremath{\F_{\msf{sync},#1,#2}}}
\newcommand{\Fsync}[2]{\ensuremath{\F_{\msf{BD-SEC}}(#1,#2)}}
\newcommand{\Fchan}[2]{\ensuremath{\F_{\msf{sync-chan}}(#1,#2)}}
\newcommand{\Fbdsec}{\ensuremath{\F_{\msf{BD-SEC}}^{\delta,\ell}}}
\newcommand{\Fbc}{\ensuremath{\F_{\msf{broadcast}}}}
\newcommand{\Fsfe}{\ensuremath{\F_{\msf{SFE}}}}
\newcommand{\Fstate}{\ensuremath{\F_{\msf{state}}}}
\newcommand{\Fclock}{\ensuremath{\F_{\msf{clock}}}}
\newcommand{\Fpay}{\ensuremath{\F_{\msf{pay}}}}
\newcommand{\Gledger}{\ensuremath{\f{G}_{\msf{ledger}}}}
\newcommand{\Wsync}{\ensuremath{\mathcal{W}_{sync}}}
\newcommand{\dealer}{\ensuremath{\mathcal{D}}}
\newcommand{\globalf}[1]{\ensuremath{{\overline{\mathcal{#1}}}}}
\newcommand{\todo}[1]{\textcolor{red}{#1}}
\newcommand{\edict}{\{\}}
\newcommand{\lar}{\leftarrow}
\newcommand{\rar}{\rightarrow}
\newcommand{\Init}{{\bf \color{NavyBlue} Init}~}
\newcommand{\OnInput}{{\bf \color{Black} On input}~}
\newcommand{\Allinputs}{{\bf \color{Cerulean} All other input~}}
\newcommand{\OnAdvInput}{{\bf \color{BrickRed} On input}~}

\makeatletter
\newcommand{\inmsg}[1]{%
	(#1\checknextarg}
\newcommand{\checknextarg}{\@ifnextchar\bgroup{\gobblenextarg}{)~}}
\newcommand{\gobblenextarg}[1]{, #1\@ifnextchar\bgroup{\gobblenextarg}{)~}}
\makeatother

\newcommand{\transfermsg}{\inmsg{transfer}{to}{val}{data}{from}}
\newcommand{\createmsg}{\inmsg{contract \ create}{addr}{val}{data}{private}{from}}
\newcommand{\reject}{\textbf{reject}~}
\newcommand{\ignore}{\textbf{ignore}~}
\newcommand{\For}{\textbf{For}~}
\newcommand{\Env}{\ensuremath{\mathcal{Z}}}
\newcommand{\While}{\textbf{While}~}
\newcommand{\Buffer}{\textbf{Buffer}~}
\newcommand{\Send}{\emph{Send}~}
\newcommand{\Output}{\emph{Output}~}
\newcommand{\Leak}{\textbf{Leak}~}
\newcommand{\In}{\textbf{in}~}
\newcommand{\If}{\textbf{If}~}
\newcommand{\Else}{\textbf{Else}~}
\newcommand{\Return}{\textbf{Return}~}
\newcommand{\pluseq}{\ensuremath{\mathrel{+}=}}
\newcommand{\Adv}{\ensuremath{\mathcal{A}}}
%\newcommand{\Partyi}{\ensuremath{\mathbf{P_i=(sid,pid)}}}
\newcommand{\Partyi}{\ensuremath{p_i}}
\newcommand{\dquad}{\quad \quad}
\newcommand{\qqquad}{\qquad \quad}
\newcommand{\qqqquad}{\qqquad \quad}
\newcommand{\qqqqquad}{\qqqquad \quad}

\newcommand*\circled[1]{\tikz[baseline=(char.base)]{
            \node[shape=circle,draw,inner sep=1pt] (char) {#1};}}

\newcommand*\token{~\circled{t}}



\begin{document}

\pagestyle{fancy}
\pagenumbering{arabic}

\maketitle

\begin{abstract}
Universal composability (UC) is a framework to define and analyze the security of cryptographic and distributed protocols.
The notion of security afforded by UC ensures that security of protocols is guaranteed in any environment, i.e. while interacting with any other UC-defined protocol. 
UC definitions of protocols enable highly modular design that greatly improves the complexity of protocols and the ease with which security properties can be proven.
Unfortunately, UC definitions and security proofs are often long, complex and difficult to reuse. 
Furthermore, disproving UC proofs and finding counter examples becomes very difficult as the complexity of the protocol at hand increases as well.
\todo{something, something engineering, software}

In this work we attempted to bridge the gap and create an implementation of UC that can be used to implement real protocols in the UC-style: taking advantage of the modularity it provides.
An impementation of UC also makes discovering and findind bugs easier through existing techniques like fuzz testing and executable definitions.
We implement UC in Python and define synchronous protocols in the model proposed by Katz et al.~\cite{katz-clock}. 
We work our way up from a simple broadcast primitive~\cite{bracha-broadcast} to distributed blockchain applications.
\end{abstract}

\section{Intro}

\section{Related Works}
\subsection{Communication Models}
The basic UC framework operates in the asynchronous communication model.
Every pair of parties, $p_i,p_j$, has analogous channels, $\mu_{[i,j]},\mu_{[j,i]}$.
Each channel accepts messages from the sender and leaks its contents to the adversary.
The message is then only delivered when the adversary writes \texttt{ok} to the channel.
Unlike other models of asynchrnouous communication where \emph{eventual delivery} is guaranteed, in this model the messages may never be delivered by the adversary.
The functionality is shown in Figure~\ref{fig:uc_channel}.

\begin{figure}[h]
	\begin{bbox}[title={Functionality $\F_{channel}(p_r,p_s)$}]

\OnInput \inmsg{m} from $p_r$:
	\begin{renumerate}
		\item Store $m$ and leak $m \rightarrow \mathcal{A}$
	\end{renumerate}
	
\OnInput \inmsg{\texttt{ok}} from $\mathcal{A}$:
	\begin{renumerate}
		\item Send $m \rightarrow p_s$
	\end{renumerate}
\end{bbox}

	\caption{The asynchronous channel defined in UC framework~\cite{uc}}
	\label{fig:uc_channel}
\end{figure}

In this work we demonstrate wrapper for both asynchronous and synchronous communication.
The most notable other synchronoud UC model is the one presented by Katz et al.~\cite{katz-clock}.

\clearpage

Nodes on \cite{kiayias2016fair}:

\begin{itemize}
    \item This model starts with describing how to model execution of \emph{synchronous} protocols that can access a global setup clock.
    \item In a previous treatment, the clock in UC was local to each party and it would have to receive update messages from the other parties (everyone is doing this operation). Hence, with GUC the environment can control the clock speed and define when clock updates happen (as other protocol sessions might also be accessing it).
\end{itemize}

There are several works from the past few years that try to model a blockchain within the Universal Composability framework---some attempting to model it in its extendion, (G)UC \cite{uc, guc}.

% Modelling blockchain for reward/penalty in a fair MPC setting, downfalls reported in next paragraph
Kiayias et al.~\cite{kiayias2016fair} models a Bitcoin-like blockchain for fair and robust multi-party computation.
It is motivated by the impossibility result for fairness in secure MPC \footnote{Fairness in MPC is defined as: either all parties learn the output or none of them do.} and circumeventing it by imposing monetary penalties on participants.
The model consists of two global functionalities, $\globalf{G}_{\msf{clock}}$ and $\globalf{G}_{\msf{blockchain}}$.
The blockchain functionality enables the expected functionality like submitting tranasctions, validating them, batching them into blocks, and allowing an adversary to reorder transcations.
Because of the GUC framework, the state of the blockchain is available to all parties including the environment and any other protocol sessions (or dummy parties).
This work however, fails to prove that their model of the blockchain is GUC-realized in any currently existing blockchain system.
Such a security proof is essential as it provide credibility to the possibility of implementing protocols in the $\globalf{G}_\msf{blockchain}$-hybrid world.
Furthermore, the assumptions that are made for the blockchain and what the adversary can do severly limit the scope of adversaries in the rearl-world.
The first failure of this model is to consider an adversary which can change the view some parties have of the blockchain state.
For example, if the adversary mines a new block and keeps it a secret, or if some nodes have not received new blocks because of communication delays.
Another failure is that all transactions in the buffer between blocks are always included in the next block.
This, again, prevents a miner-like adversary which can censor transactions and delay their entry into the chain.
Finally, the state of the blockchain is updated at fixed time intervals which does not accurately convey the consensus model of Bitcoin or Ethereum.

% Bitcoin composable treatment in GUC
Badertscher et al.~\cite{badertscher2017bitcoin} attempt to solve these problems by allowing a more unrestricted in the GUC framework.
The shared functionality in this case is a global clock functionality, $\globalf{G}_\msf{clock}$, which enables modelling a synchronous system in the UC framework by proceeding in rounds.
Because it is a shared functionality, the clock allows any other protocol session in the environment to be synchronized with the challenge protocol. 
The blockchain functionality is a local functionality (only available to the parties within the protocol session) that allows the adversary to have more power in what it can do.
The adversary can inject transactions and modify the state of the chain that all parties that query it can see.
This is accomplished by allowing a maximum distance, $d$, that the adversary can specify and return a prefix of the chain which is at most a distance $d$ from the head of the chain.
Furthermore, the adversary can choose exactly which transactions are allowed to be in the next block.
The blockchain functionality is modularized by allowing the definition of subroutines that capture extending the blockchain state (specifically for Bitcoin in this paper).
The authors of this work admit that the paper's only intent is to model the Bitcoin blockchain hence the choice to use the ledger as only a local functionality. 
This prevents other protocol sessions from using the same blockchain (definitely a limitation of modelling the reality of a blockchain environment).
Furthermore, this paper makes the argument that it is dangerous to have a global ledger functionality as such replacement does not ``in general, preserve a realization proof of some ideal functionality $\F{}$ that is conducted in a ledger-hybrid world, because the simulator in that proof might rely on specific capabilities that are not available any more after the replacement (as the global setup is also replaced in the real world)''.
It claims that~\cite{canetti2016universally} provides a sufficient condition for such a replacement, but that the condition is too strong to be satisfied by any ledger implementation.

% UC with a global PKI
Canneti et al.~\cite{uc-pki} addresses the global PKI and an ideal authentication within the UC with global setup.
The specific problem presented in this paper is that the ideal authentication functionality, $\F_\msf{auth}$,is usually formulated with the desirable property of non-transferrability of authentication.
This means that when I send an authenticated message to another person, they are unable to use that proof to convince anyone else of the authentication.
The paper realized that the real world PKI model is global \emph{and} that, within it, signatures are globally verifiable.
Once a key has signed a message for authentication, that proof is verifiable by and transferrable to anyone else in the system.
Therefore, this work models a new relaxed global PKI, relaxes the UC authentication protocol to not require deniability, and formulates new functionalities for authentication and key exchange without deniability.
Finally, they propose a new composition theorem allowing substitution of global functionalitites, $\f{F} \textit{ EUC-realizes } \f{G}$.
The problem being solved relates back to a claim made by Badertscher et al.~\cite{badertscher2017bitcoin} that replacement of global functionalities with real implementations generally invalidates a realization proof of some functionality that shares state with it. 
In this paper, this arises as replacement of the UC PKI system with a real one where transferrability is possible invalidates the realization proof of the ideal authentication functionality in the plain-PKI model.

They formulate a new authentication functionality that does not impose non-transferrability and a long lasting global functionality handling certificates.
Finally they prove that the certificate functionality guarantees are precisely captured by EU-CMA signatures and a globally-available PKI .
This paper however imposes some restrictions on what can be done.
For example, there is a limitation that a particular ITI may only register a single key with the Cert and Bulletin Board functionalities.
They claim however, that it is possible to realize $\F_\msf{cert\_auth}$, but a certificate-based approach is not it.

One of the main takeaways in this paper is that you can define a functionality and analyze it for it's properties then prove that it is equivalent to another functionality that realizes this protocol. 
In this paper that is done by defining

\paragraph{Differentiating $\f{G}_\msf{cert}^\msf{pid}$ and $\f{G}_\msf{swk}^\msf{pid}$.}




Questions to answer:
\begin{itemize}
\item What is the precise difference between $\f{G}_\msf{cert}^\msf{pid}$ and $\f{G}_\msf{cwk}^\msf{pid}$ and why is the substitution necessary?
\end{itemize}




\section{Preliminaries}
To build up to a ledger functionality, we first need to discuss the building blocks.
The first important component is the communication model.
Recall from~\cite{uc} that the communication model is asynchronous where messages between parties can be arbitrarily delayed by the adversary.
Though this is the weakest assumptions that one can make about a network, we require a synchronous communication model build on top of UC that can provide some eventual delivery guarantees.

\subsection{Synchronous Network}
There are two parts that go into modelling a synchronous network in UC: creating a round structure that all ITMs can be synchronized with and requiring maximum delays on message by the adversary.
In order to achieve the former, we rely on a pervious work by Katz et al.~\cite{katz-clock}.



\section{Import Model}
One of the goals of this work, alongside the new models for synchronous and asynchronous communication is to present an impementation of the UC framework.
One of the goals of this implementation is to bridge the gap between the paper proofs in the UC framework and actual realizations of protocols in a real programming environment.
Another is to provide an easy platform through which UC definitions and proofs can be more closely scrutinized with executable counter examples.
In order to achieve this goal, we need to first more carefully define how the import mechanism~\cite{uc} workd.
The previous section described the mechanism to the extent that it is presented in the original work by Canetti et al.~\cite{uc}.
However, this definition lacks specific model for how import is handled at the ITM and message level.
In this section we present the model of computation presented in the original UC framework but augmented with import.

The import mechanism most resembles a token system where a finite number of tokens is distributed throughout the system.
We will use the token terminology throughout the remainder of this paper.
Before introducing the model, we first make a distinction between {\em import tokens} and {\em potential}.
As mentioned above import tokens are in finite supply and can be transferred between ITMs.
However, import is consumed by an ITM through a polynomial $T$.
If an ITM, $m$, with zero import tokens receives a message of $n$ tokens, this ITM can now take $T(n)$ computationsl steps.
We refer to $T(n)$ as the {\em potential} of the ITM\footnote{It is important to note that the ITM $m$, must consume an entire import token if it is to use even one unit of potential.}.

The term ``computational steps'' also warrants further clarification.
So far we are keeping the exact definition of a single computational step vague on purpose.
We posit that as long as a single computational step actually corresponds to a constant amount of work, the remainder of the model still holds. \todo{this is very unsatisfying, need to make a clearer statement about definitions of a single ``computation step''}


The first addition to the ITMs defined in \cite{uc} is a new tape called the \msf{import\ tape}.
This tape consists of the following:
\begin{itemize}
	\item An integer, \msf{input\ import}, that tracks the total number of import tokens ever sent to the ITM on any of its externally writeable tapes.
	\item The \msf{output\ import} tracks the total number of tokens ever sent to any other ITM through its extranlly writeable tapes.
	\item The \msf{spent\ potential} is the total amount of potential consumed by the ITM.
	\item The \msf{marked\ tokens} is the total number of tokens that are consumed and converted into potential through the polynomial $T$.
\end{itemize}

-- Change made to the external-write command to include the amount of import sent with the message

-- state that computation steps here is an imprecise term and it doesn't really matter how it's defined, but any reasonable description of a computation step should still work. The main thing is constant vs linear vs polynomial work which must be captured in the definition

-- when ITM does computation steps (need to formalize this some more because what exactly is a computation step in ITMs versus python) ticks is issued, and potential is generated when potential runs out but more computation is needed. Perhaps the right way to do it is to have tick called at the start of every activation for the amount og 

-- ok so restate the ITM definitions from the UC paper. Stress that they do not interfere with the default definition so no need to reprove anything (import = 0 everywhere woul yield an equivalent ITM configuration).

-- describing how ``tick'' is used might be more than can be said right now (might require some more formality).

-- the state the definition/claim/lemma that an ITM that follows these rules will be T-bounded then state the entire execution is T-bounded because all the subsidiaries are T-bounded (use the argument from the UC paper about subroutines).


\paragraph{Cost Model}
The cost model defines the import cost of primitive operations that an ITM can perform in our proposed realization of the import mechanism.
We define a protocol or implementation that adheres to the cost model and the import mechanism above achieves T-bounded-ness in its UC experiments.
In the implementation of UC, programs self report the cost of their computations. 
We propose that a program that follows the cost model in our implementation does achieve T-boundedness.
Therefore, adversaries that don't report their computation steps through the cost model are not valid and therefore not guaranteed to comply with our import statement.


need a cost model to represent basic operations that an ITM can perform and assign tick values to them

primitive operations are write/read message to/from another ITM and create a new ITM (spawn a new one by calling it)

The cost of the parameters of the create new ITM are already paid for when computing their amounts so creating a new ITM is a constant work operation

The cost of sending a message though is dependent on the size of the message

Define this cost model and an ITM system that implements this cost model has these properties that it satisfied T-bounded ness as is required by the import mechanism



ITMs also have a new instructions
\begin{itemize}
	\item \msf{generatepot} consumes some number of tokens and converts than into usable potential. This function is usually only used when the ITM is out of potential and wants to generate more. If the ITM isout of import, this function fails and the ITM doesn't proceed with any further computation setps. Instead is returns control to \mc{Z}.
	\item 
\end{itemize}



\section{Asynchronous Communication} \label{sec:async}
In this section we introduce our models for asynchronous and synchronous communcation.
Both are implemented through wrappers, $\mathcal{W}_{\msf{sync}}$ and $\mathcal{W}_{\msf{async}}$, which use the new {\em import mechanism} introduced in ~\cite{uc} and described in detail in Section~\ref{sec:prelim}.
Before defining our wrappers we first define how ITMs leak information to the adversary.

\subsection{Leakage Wrapper}
In the original UC framework, leakage of information to the adversary means writing a message to the backdoor tape of $\mathcal{A}$.
Leaking like this immediately passes control to the adversary an the leaking ITM can not take any further action. 
As we describe below, ITMs often leak information {\em and} asynchronously execute code in the same activation, and it is not possible to do so when the adversary gains control.
Instead, we use a wrapper called the {\em leakage wrapper} which stores all messages leaked from all parties and all functionalities.
The wrapper stores the leaked msg and the import sent with it, and it returns control back to the calling ITM.
Figure \ref{fig:wrapper:leak} describes how the wrapper works.
For the sake of simplicity, we combine the leakage code with the communication wrappers to create one ITM that performs all the requisite tasks.
\begin{figure}[h]
	\begin{bbox}[title={\textbf{Wrapper} $\mathcal{W}_{\msf{leak}}$} ]

Initiaize $\msf{leaks} := \emptyset$

\vspace{2mm} \hrule \vspace{2mm}

\OnInput \inmsg{leak, msg} $d \token$ from $\F_i$/$P_i$:

	\begin{ritemize}
		\item Append $(msg, d \token)$ to \msf{leakbuffer}
		\item Send \msf{(OK)} back to the caller $\F_i$/$P_i$
	\end{ritemize}

\OnInput \inmsg{getleaks} from $\mathcal{A}$:

	\begin{ritemize}
		\item Pop all $(msg_i, d_i \token)$ and send $\{msg_1,...,msg_n\}, \sum_{i=0}^{n} d_i \token$ to $\mathcal{A}$
	\end{ritemize}

\end{bbox}

	\caption{The leakage wrapper buffers all leaked messages and their import until the adversary requests them with \texttt{getleaks}. \mc{A} is sent all the leaks and import when requesting.}
	\label{fig:wrapper:leak}
\end{figure}

For the remainder of the paper we make a distinction between {\em sending} a message to \mc{A} and {\em leaking} information to \mc{A}.
For example, when \mc{Z} sends an \texttt{advance} message to the wrapper, the wrapper directly activates \mc{A} with an \texttt{advance} message.
As a consequence of buffering all leaked messages, the adversary must always attmpted to retrieve leaks from the wrapper at every activation.
In the real world with a dummy adversary the environment can request leaks as needed, however simulators in the ideal world must make sure to always request new leaks on every activation.

\subsection{Asynchronous Wrapper}
Our asynchonrous communication model consists of a wrapper that exists in both the real and ideal worlds.
Unlike the works of Katz et al.~\cite{katz-clock} and Coretti et al.~\cite{coretti}, our model enables the asynchronous execution of entire codeblocks rather than just message delivery.
Both of the previous works rely on the adversary using a unary message format for delaying the delivery of messages.
Although this is sufficient to achieve {\em eventual delivery}\footnote{Unary format enforces a polynomial amount of delay after which the messages can not be delayed further.}, it forces protocols and funtionalities to indepdendently deal with adversarial delay and requires activations in order for messages to be {\em fetched} as in \cite{coretti}.
In our formulation, an ITM can simply mark a codeblock to be executed asynchronously, and the wrapper takes care of enforcing its delivery with the use of the new {\em import} mechanism presented in \cite{uc}.
In doing so, the wrapper abstraction we present here greatly removes communication model-specific artifacts from the code of functionalities and protocols. 
Furthermore, as we will see in both the asynchronous and synchronous wrappers, the abstraction enables functionalities that are uniform accross the two models.
The program for both functionality and protocol in the example that we present, an atomic broadcast protocol, do not depend on the communication model at play\footnote{The only difference between the two is that the synchronous wrapper accepts an additional parameter, $\Delta$, for the upper bound on round delay.}.

\begin{figure}
\centering
	\begin{bbox}[title={\textbf{Wrapper} $\mathcal{W}_{\msf{async}}$} ]

Initialize $\msf{leakbuffer} := \emptyset, \msf{runqueue} := \emptyset, delay := 0$

\vspace{2mm} \hrule \vspace{2mm}

\OnInput \inmsg{schedule}{codeblock $e$} $1 \token$ from $\mathcal{F}_i$/$P_i$:
	\begin{ritemize}

		\item Append $e$ to $\msf{runqueue}$ with index $idx$, and increment $delay \pluseq 1$
		\item Append (schedule, $idx$) to \msf{leakbuffer} and respond to $\F_i,P_i$ with \msf{(scheduled)}

	\end{ritemize}

\OnInput \inmsg{delay} $d \token$ from $\mathcal{A}$:
	\begin{ritemize}
		\item $delay = delay + d$
	\end{ritemize}

\OnInput \inmsg{execute}{$idx$} from $\mathcal{A}$:
	\begin{ritemize}	
		\item Pop a codeblock off $\msf{runqueue}$  and execute it.
	\end{ritemize} 

\OnInput \inmsg{leak}{msg} from $\mathcal{F}_i/P_i$:
	\begin{ritemize}
		\item Append $(msg, \mathcal{F}_i/P_i, d \token)$ to $\msf{leakbuffer}$
	\end{ritemize}

\OnInput \inmsg{advance} $1 \token$ from $\mathcal{Z}$:
	\begin{ritemize}
		\item \If $delay > 0$:
			Decrement $delay \minuseq 1$ and leak \msf{(poll)} to $\mathcal{A}$
		\item \Else: 
			Pope a codeblok off $\msf{runqueue}$ and execute it.
	\end{ritemize}

\OnInput \inmsg{getleaks} from $\mathcal{A}$:
	\begin{ritemize}
		\item Sum the total amount of import in $\msf{leakbuffer}$, call it $d$.
		\item Leak $(\msf{leakbuffer},d \token)$ to $\mathcal{A}$ and empty $\msf{leakbuffer}$
	\end{ritemize}
\end{bbox}

	\caption{The asynchronous wrapper. It accepts codeblocks from functionalities and parties, and it allows the adversary to decide execution with finite delay. The environment is also able to force progress in the protocol.}
	\label{fig:wrapper:async}
\end{figure}

The asynchronous wrapper in Figure~\ref{fig:wrapper:async}, maintains a queue of codeblocks and a delay parameter representing the amount of delay that the adversary has assigned before the {\em next} codeblock is executed. 
When a codeblock is added to the wrapper, the delay is incremented and information is leaked\footnote{Recall, to ``leak'' in this context means to append the message to the buffer of leaks in $\mc{W}_{\msf{leak}}$ to the adversary.} to \mc{A} about the scheduling ITM and the the codeblock's location in the queue.
The adversary can add delay to the wrapper whenver it wants to, and can execute any codeblock in the \msf{runqueue} whenever it wants.
Adding to the delay, however, requires the adverarsy to send 1 unit if import to the wrapper\footnote{Recall that the import mechanism was created in order to enforce polynomial run time in the whole system of ITMs spawned in the execution of a UC experiment.}.
\todo{resolve import-delay issue brought up by shreyas}.

\paragraph{Protocols in the $\{\mathcal{W}_{\msf{async}}, \F_{\msf{achan}}^{p_r,p_s}-hybrid\}$ world.}
Protocols in the real world have access to one-shot asynchronous channel functionalities, \achan, parameterized by a sender and a receiver.
\achan is shown in Figure ~\ref{fig:func:achan}


\begin{itemize}
\item The adversary has complete control over the order in which the codeblocks are executed. 
It can send 1 unit of import and an \texttt{exec} message to the wrapper to pop any current codeblock off the queue and execute it. 
\item As we're concerned with {\em eventual delivery}, it is insufficient to only allow the adversary to execute codeblocks and determine their delivery. 
Therefore the environment can also force the wrapper to make progress by spending 1 unit of import to \texttt{advance} the wrapper.
Over multiple such calls, the environment forces the delay to be 0, at which point the {\em next} codeblock in the \msf{runqueue} is removed and executed.
\end{itemize}

A critical point to address is {\em how} the environment can force the delay to reach 0.
Even though the adversary can provide delay to the wrapper, it only has a limited amount of import that it can spend to do so~\footnote{Recall from Section \ref{sec:prelim} that a {\em balanced environment} must provide the the adversary with at least as much import as it gives to each of the other ITMs. Therefore, a simulator has finite import but enough to simulated a sandboxed real-world execution.}.
Therefore, the adversary will eventually run out of import and lose the ability to delay the wrapper furter.
Therefore, the bound on the delivery of the messages is unknown but upper-bounded by the polynomial runtime guaranteed by a finite amount of import provided to the adversary.


%The model presented in this paper for both asynchronous and synchronous communication relies on the new \emph{import} mechanism described in \ref{uc} and the wrappers we present below.
%The first departure that our wrapper make from tradition UC communication models is that they execute arbitrary code blocks rather than just deliver messages.
%For example, a complex application may require some loging to execute some sort of state update function at a future time dependent on the communication model.
%In the synchronous world, this may mean executing a codeblock within some number of rounds.
%Similarly, the asynchronous world allows {\em eventual} execution of codeblocks.
%In this section, however, we focus only on the asynchronous wrapper as its functionality is nearly identical to that of the synchronous wrapper minus some logic to ensure {\em guaranteed termination} and {\em input completeness}.

The asynchronous wrapper provides an interface to functionalities and protocol parties to execute codeblocks asynchronously.
The wrapper is shown in Figure~\ref{fig:wrapper:async}.

\begin{itemize}
\item describe the wrapper and how the import mechanism is used to ensure eventual delivery.
\item Designing protocols with the wrapper
\end{itemize}


\section{Synchronous Communication} \label{sec:sync}
\subsection{Broadcast in Katz et al.}

In this section we describe the ideal functionality and protocol for a simple byzantine broadcast in the synchronous model.
The function is modelled after the secure function evaluation functionality, $\F_{\msf{SFE}}^{f,Rnd}$, described in ~\cite{katz-clock}.
Secure function evaluation computes the output of a single function from the inputs of the parties.
The function being evaluated must be computable in one round in the ideal world, meaning parties don't have to give input more than once.
In our case, the ideal function for a byantine broadcast, $f$ (shown below), with a dealier, $\mathcal{D}$, just selects the dealer's input.

\[ f_{bc}(x_1,...,x_n) := x_{\mathcal{D}} \]

The functionality proceeds in rounds, $l$, until $Rnd$ rounds have elapsed.
At the end of $Rnd$ rounds it computes the function $f$ on the inputs of all parties that have provided input and sends output to each of them.
In every round, the functionality requires the environment to activate each party at least $|\mathcal{P}|$ times before increenting the round $l$. 
Everytime a party is activated by a party asking for \msf{output}, the functionality activates the the simulator.
With $n$ activations, the simulator can sufficiently simulate the real world protocol and use the activations provdided by the functionality, to activate and perform computation.
For example, for a party $p_i$, on activation $k$ within a round, the simulator simulates $p_i$'s interaction with party $p_k$ in the real world.
It will read messages from $p_k$ to $p_i$, perform any local computation, and potentially send messages to $p_k$ for subsequent rounds.

Similarly, such a structure design must also be present in the real world, where the environment will provide the same $n$ \msf{output} to each itm.

\begin{figure}[!h]
	\begin{bbox}[title={$\Fbc (\mathcal{D}, \mathcal{P} = p_1,...,p_n)$}]

Intialize $x_\dealer := \bot, \ell := 1, \forall p_i : t_i = |\mathcal{P}|$

\vspace{2mm} \hrule \vspace{2mm}

-- \OnInput \inmsg{input}{$v$} from \Partyi:
	
	\qquad Set $x_\dealer := v$

	\qquad \Leak $v$ to $\mathcal{A}$

	\qquad Within $O(Rnd)$ rounds, deliver $f(x_1,...,x_n)$ to $\forall p_i$

%-- \OnInput \inmsg{output} from \Partyi:
%
%	
%	\qquad \If ($p_i = \dealer$) and ($x_\dealer$ not set): ignore 
%
%	\qquad \Else \If $(t_i > 0)$: Set $t_i := t_i - 1$
%
%		\qquad \quad \If $(\forall p_i \in \mathcal{H})$: Set $\ell := \ell + 1$
%
%	\qquad \Else \If $(t_i = 0)$ and $(\ell < Rnd)$: \Send (early) $\rightarrow p_i$
%
%	\qquad \Else \If $(y_1,...,y_n)$ not set:
%
%		\qquad \quad Set $y_1,...,y_n := x_\dealer$
%
\end{bbox}


	\label{fig:functionality:broadcast_high}
	\caption{The ideal functionality for the synchronous Byzantine broadcast protocol from~\cite{bracha}. This functionality abstracts away UC-related details for clarity. This description of the functionality will actually be compiled to the functionality below which is how the program be written to adhere to the UC framework.}
\end{figure}

\begin{figure}[!h]
	%\begin{bbox}[title={Wrapper $\mathcal{W}_{\msf{In-O(1)}} (\F)$}]
%
%Initialize $\msf{crnd} := 0$, $\msf{lastcrnd} := -1$, $\msf{runqueue} := []$
%
%\vspace{2mm} \hrule \vspace{2mm}
%
%\OnInput \inmsg{In-O(1)}{codeblock e} from $\F$:
%
%	\quad Add $e$ to $\msf{runqueue}$
%
%	\quad \Leak $e \rightarrow \mathcal{A}$
%
%\OnInput \inmsg{deliver}{idx} from $\mathcal{A}$
%
%	\quad $e \leftarrow \msf{runqueue}[idx]$
%
%	\quad Delete $\msf{runqueue}[idx]$
%
%	\quad {\bf Execute} $e$
%
%\vspace{2mm} \hrule \vspace{2mm}
%
%On every activation:
%
%	\quad $\msf{rnd} \leftarrow \F_{\msf{clock}}.\msf{clockread}$
%
%	\quad \If $\msf{rnd} \neq \msf{crnd}$:
%
%		\quad \quad $\msf{lastcrnd} \leftarrow \msf{crnd}$
%
%		\quad \quad $\msf{crnd} \leftarrow \msf{rnd}$
%
%\end{bbox}

%\begin{bbox}[title={Wrapper $\mathcal{W}_{\msf{O(1)}}$}]
%
%Initialize $\msf{crnd} := 0$, $\msf{lastcrn} := -1$, $\msf{runqueue} := []$
%
%\vspace{2mm} \hrule \vspace{2mm}
%
%\OnInput \inmsg{In O(1)}{codeblock e} from $\F$:
%
%	\quad Add $e$ to $\msf{runqueue}[\msf{crnd}+1]$ 
%
%\OnInput \inmsg{deliver}{idx} from $\mathcal{A}$:
%
%	\quad Pop $e \leftarrow \msf{runqueue}[\msf{crnd}][idx]$
%
%	\quad Execute $e$
%
%\end{bbox}

\begin{bbox}[title={$\F_{\msf{Bracha}} (\mathcal{D}, \mathcal{P} = p_1,...,p_n)$}]

See $\F_{\msf{SFE}}^{f,Rnd}$ in Katz.
%\OnInput \inmsg{input}{T} from $\mathcal{D}$ (once):
%
%	\quad \If $\msf{crnd} > 0$: \reject
%
%	\quad \Leak $T \rightarrow \mathcal{A}$
%
%	\quad \For $p_i \in \mathcal{P}$:
%
%		\quad \quad {\em In O(1)}  \Send $T \rightarrow p_i$
%
%\vspace{2mm} \hrule \vspace{2mm}
%
%\If $\msf{crnd} > 0$ and no input from $\mathcal{D}$:
%
%	\quad \For $p_i \in \mathcal{P}$:
%
%		\quad \quad {\em In O(1)} \Send $\bot \rightarrow p_i$
%
\end{bbox}

%\begin{bbox}[title={Simulator $S_{\msf{Bracha}}$}]
%
%Simulate real-world parties $\overline{\mathcal{P}} = p_1,...,p_n$ and $\Fsync{p_i}{p_j}, \forall p_i,p_j \in \overline{\mathcal{P}}$
%
%Simulate instance $\overline{\F}$ of $\F_{\msf{clock}}$.
%
%Designate same dealer $\overline{\mathcal{D}}$ as environment.
%
%Simulate dummy adversray $\mathcal{A}_{\mathcal{D}}$
%
%\vspace{2mm} \hrule \vspace{2mm}
%
%Case \#1 ( Dishonest $\mathcal{D}$ ):
%
%\OnInput \inmsg{input}{v} from $\mathcal{Z}$ for $\mathcal{D}$:
%
%	\quad \Send (input,v) $\rightarrow \mathcal{A}_{\mathcal{D}}$ {\em (Passthrough for corrupted parties in real world)}
%
%\OnInput \inmsg{m} from $\mathcal{Z}$:
%
%	\quad \Send (m) $\rightarrow \mathcal{A}_{\mathcal{D}}$
%
%\OnInput \inmsg{activates}{$p_j$} from $\F_{\msf{Bracha}}$:
%
%	\quad \If first message in round $r$:
%
%		\quad \quad Deliver messages from $\Fsync{p_j}{p_i}$ to $p_i$ through $(\msf{fetch})$ and simulate state changes.
%
%\vspace{2mm} \hrule \vspace{2mm}
%
%When protocol terminates, obtain output value $v$. Deliver $v \rightarrow \F_{\msf{Bracha}}$ as the dealer $\mathcal{D}$.
%
%\end{bbox}
%\begin{bbox}[title={Simualator $S_{\msf{Bracha}}$}]
%
%Simulate real-world parties $\mathcal{\overline{P}} = p_1,..,p_n$ anbd $\Fsync{p_i}{p_j}, \forall p_i,p_j \in \mathcal{P}$ and corrupt $t$ of them.
%
%Simulate instance $\overline{\F}$ of $\F_{\msf{clock}}$ and instance $\overline{\mathcal{W}}$ of wrapper $\mathcal{W}_{O(1)}$.
%
%Designate the same dealer $\overline{\mathcal{D}}$ as the ideal protocol.
%
%Simulate the real world adversary $\mathcal{A}$
%
%\vspace{2mm} \hrule \vspace{2mm}
%
%\OnInput \inmsg{T} from $\F_{\msf{Bracha}}$ \emph{(input to $\F_{\msf{Bracha}}$ from $\overline{\mathcal{D}}$)}:
%
%	\quad Submit $T$ to $\overline{\mathcal{D}}$
%
%	\quad Simulate state changes in all praties until $\overline{\F}.\msf{round}$ increments
%
%\OnInput \inmsg{deliver}{idx} from $\mathcal{Z}$:
%
%	\quad \Send (deliver,idx) $\rightarrow$ $\overline{\mathcal{W}}$
%
%\OnInput \inmsg{clockupdate}{$p_i$} from $\mathcal{Z}$:
%
%	\quad \If $p_i$ is corrupted: \Send (clockupdate) $\rightarrow \overline{\F}$
%
%	
%
%\end{bbox}

	\label{fig:functionality:broadcast}
	\caption{Ideal functionality for a byzantine broadcast. The ideal functionality is not specific to any potential real world protocol, but captures the guarantees of the broadcast. Additionally, the functionality is identical to the SFE functionality except for assumptions made for the application at hand.}
\end{figure}

The ideal functionality of for byzantine broadcast is described in Figure ~\ref{fig:functionality:broadcast}
The functionality is cast as a secure function evaluation where the function is as described above.
In our case we simplify the SFE functionality instead of just parmeterizing it with the function $f_{\msf{bc}}$ gives its simplicity.
The only simplification made to SFE is that the functionality only waits for the dealer's input (in SFE this equates to assuming all other parties' inputs are already set).
Furthermore, the function $f_{\msf{BC}}$ is put in place in the functionality.

\begin{figure}
	\begin{bbox}[title={\textbf{Protocol} Async-Bracha-Broadcast$(v)$}]

-- {\bf Step 0}. Input $v$ from $p$:
	
	\qquad Send $(initial,v)$ to all processes.

-- {\bf Step 1}. Wait for:
	
	\qquad one $(initial, v)$ message or $\frac{n+t}{2}$ $(echo,v)$ messages or $(t+1)$ $(ready,v)$ messages, for some $v$.

	\qquad Send $(echo,v)$ to all processes.

-- {\bf Step 2}. Wait for:
	
	\qquad $\frac{n+t}{2}$ $(echo,v)$ messages or $(t+1)$ $(ready,v)$ messages, for some $v$

	\qquad Send $(ready,v)$ to all processes.

-- {\bf Step 3}. Wait for:

	\qquad $2t+1$ $(ready,v)$ messages

	\qquad Accept $v$.

\end{bbox}

	\label{fig:protocol:asyncbracha}
	\caption{The original asynchronous broadcast protocol proposed by Bracha~\cite{bracha-broadcast}. The protocol proceeds in three rounds and terminates if enough messages are delivered.}
\end{figure}

Next we introduce a real world protocol that realized the ideal functionality.
We use the asynchronous broadcast primitive introduced by Bracha~\cite{bracha-broadcast} that tolerates $\frac{n}{3}$ byzantine failures.
Below, we introduce the same broadcast protocol cast in the synchronous model.
We modify the protocol in the following ways:

\begin{itemize}
	\item First, the protocol has to be updates to reflect the interace offered by the ideal functionality. In the ideal world, the environment gives $|\mathcal{P}|$ activations to each party in each round. In the real world protocol, on the first activation of a party in a round $r$, the party $p_i$ fetches incoming messages from previous rounds, computes the messages to be sent in this round, $\{m_{i,j,r}\}_{j}$, and uses the next $|\mathcal{P}|$ rounds to send those messages to the other parties.
	\item Second, the protocol is modified to only accept certain messages in certain rounds. For example, the hones dealer will only accept $(input,v)$ from the environment in the first round, all parties will only accept $(VAL,v)$ messages in the second tound, and so on.
\end{itemize}

\begin{figure}[!h]
	%\begin{bbox}[title={$\Pi_{\msf{Bracha}} (\mathcal{D}, \mathcal{P} = p_1,...,p_n)$ in $\F_{\msf{sync}}$-hybrid}]
\begin{bbox}[title={$\Pi_{\msf{Bracha}} (\mathcal{D}, \mathcal{P} = p_1,...,p_n)$ in $\F_{\msf{BD-SEC}}$-hybrid}]

Initialize $\msf{BQ} := \frac{\msf{ceil}(n+t)}{2}$, $\msf{init} := crnd$, $\msf{out} := \emptyset$

\vspace{2mm} \hrule \vspace{2mm}

{\bf Dealer $\mathcal{D}$ Protocol}

-- \OnInput \inmsg{input}{m} from $\mathcal{Z}$:

	\dquad \For $p_i \in \mathcal{P}$:

		\dquad \quad \Send $\msf{VAL}(m) \rightarrow \Fsync{\mathcal{D}}{p_i}$

\vspace{2mm} \hrule \vspace{2mm}

{\bf Party $p_i$ Protocol}

-- \OnInput \inmsg{$\msf{VAL}(m)$} from $\F_{\msf{sync},\mathcal{D},p_i}$ (once, round $\msf{init}+1$):

	\dquad \For $p_j \in \mathcal{P}$: \Send $\msf{ECHO}(m) \rightarrow \Fsync{p_i}{p_j}$ \\

 \OnInput \inmsg{$\msf{ECHO}(m)$} from $\Fsync{p_j}{p_i}$ (round $\msf{init}+2$):

	\dquad \If received $\msf{ECHO}(m)$ from $\msf{BQ}$ parties:

		\dquad \quad \For $p_j \in \mathcal{P}$: \Send $\msf{READY}(m) \rightarrow \Fsync{p_i}{p_j}$ \\

-- \OnInput \inmsg{$\msf{READY}(m)$} from $\Fsync{p_j}{p_i}$ (round $\msf{init}+3$):

	\dquad \If received $\msf{READY}(m)$ from $2t+1$ parties:

		\dquad \quad $\msf{out} := m$

		%\quad \quad \Output $m$

	%\quad \If received $\msf{ECHO}(m)$ from $\msf{BQ}-t$ parties and $\msf{READY}(m)$ from $> t$ parties and $\msf{READY}$ not sent:

	%	\quad \quad \For $p_j \in \mathcal{P}$: \Send $\msf{READY}(m) \rightarrow \Fsync{p_i}{p_j}$ \\

-- \OnInput \inmsg{output} from $\mathcal{Z}$:

	\dquad \If $\msf{out} \neq \emptyset$: \Output $\msf{out}$ 

	\dquad \Else On $j^{th}$ activation in this round:

		\dquad \quad \Send $(\msf{fetch}) \rightarrow \Fsync{p_j}{p_i}$

		\dquad \quad $m \leftarrow \Fsync{p_j}{p_i}$

\vspace{2mm} \hrule \vspace{2mm}

\If not received $2t + 1$ \msf{READY}(\textunderscore) messages by $\msf{init} + 4$:

	\dquad \Output $\bot$

\end{bbox}


%  RND 2: 
%			OnInput VAL from Dealer --> Echo
%  RND 3:
%			OnInput ECHO(m) from Pi: If 2t+1 ECHO messages --> READY
%  RND 4:
%			OnInput READY(m) from Pi: If t+1 READY --> Output m

\end{figure}
\begin{figure}
	\begin{bbox}[title={Simulator $S_{\msf{Bracha}}$}]

Simulate real-world parties $\overline{\mathcal{P}} = p_1,...,p_n$ and $\Fsync{p_i}{p_j}, \forall p_i,p_j \in \overline{\mathcal{P}}$

Simulate instance $\overline{\F}$ of $\F_{\msf{clock}}$.

Designate same dealer $\overline{\mathcal{D}}$ as environment.

Simulate dummy adversray $\mathcal{A}_{\mathcal{D}}$

\vspace{2mm} \hrule \vspace{2mm}

Case \#1 ( Dishonest $\mathcal{D}$ ):

\OnInput \inmsg{input}{v} from $\mathcal{Z}$ for $\mathcal{D}$:

	\quad \Send (input,v) $\rightarrow \mathcal{A}_{\mathcal{D}}$ {\em (Passthrough for corrupted parties in real world)}

\OnInput \inmsg{m} from $\mathcal{Z}$:

	\quad \Send (m) $\rightarrow \mathcal{A}_{\mathcal{D}}$

\OnInput \inmsg{activates}{$p_j$} from $\F_{\msf{Bracha}}$:

	\quad \If first message in round $r$:

		\quad \quad Deliver messages from $\Fsync{p_j}{p_i}$ to $p_i$ through $(\msf{fetch})$ and simulate state changes.

\vspace{2mm} \hrule \vspace{2mm}

When protocol terminates, obtain output value $v$. Deliver $v \rightarrow \F_{\msf{Bracha}}$ as the dealer $\mathcal{D}$.

\end{bbox}

\end{figure}

{\bf Theorem.} {\em Protocol $\Pi_{\msf{Bracha}}$ securely realized \Fbc in the $\{\Fbdsec,\Fclock \}$-hybrid world. Assume a stateic adversary corrupted up to $\frac{n}{3}$ parties.}

Consider the simulator, $\mathcal{S}$, above.

If the dealer $\mathcal{D}$ is honest: In the ideal world, $\mathcal{D}$ gives input $v$ to $\F_{\msf{Bracha}}$ which gives leaks it to $\mathcal{S}$.
The simulator submits the input to it all of the locl $\Fsync{\mathcal{D}}{p_i}$ for $p_i \in  \mathcal{P}$.

$\mathcal{S}$ expects to receive $|\mathcal{P}|$ activations from $\F_{\msf{Bracha}}$ when ideal world parties attempt to read output from the functionality.
In each activation, the simulator sufficiently ensures each party reads messages from all other parties and simualated state changes and increment the local \Fclock.

The functionality waits $Rnd = 3$ rounds to deliver the output. In the first round $|\mathcal{P}|^2$ activations ensure all \msf{ECHO} messages are sent.
In functionality round 2, activations ensure that all \msf{READY} messages are sent. The final functionality round 3, all \msf{READY}s are delivered and the simulates real world parties all output a value $v$.
By the proof of the Bracha protocol, all real world parties output the same value. The simulator instructs 

\paragraph{Typo in Katz paper}
In synchronous protocols, parties can send at most $n = |\mathcal{P}|-1$ messages, one message to each other participant in the protocol.
According to the functionality $\Fbdsec$, when a party sends a message, the adversary is activated by leaking information.
This means that in one activation, each party can only send one message.
The definition of synchronous protocols in the $\{\Fbdsec,\Fclock \}$-hybrid says that in each round each party must send (\texttt{RoundOK}) to $\Fclock$.
Therefore, each party needs $n$ activations, $n-1$ for sending messages and 1 for sending \texttt{RoundOK}.
Finally, any subsequent activation if $p_i$'s bit $d_i$ is still 1, $p_i$ outputs \texttt{early} to the environment.
The ideal world needs these many acivations as well, to invoke the simulator enough to simulate the real world.

The functionality as described in the paper, show in Figure~\ref{fig:sfe} allows the round to advance with only $n-1$ activations through the \texttt{output} message.
The edit to make is to change the \texttt{activated} and \texttt{early} logic to wait until $t_i = 0$ instead of 1. That's all.


\begin{figure}
	\begin{bbox}[title={$\F_{\msf{sfe}}^{f,Rnd} (\mathcal{P})$}]

For each $p_i \in \mathcal{P}$ initialize $x_i = y_i := \bot$, delay $t_i := |\mathcal{P}|$.
Global $\ell := 1$

-- \OnInput \inmsg{\texttt{input}}{$v$} from \Partyi:

	\qquad Set $x_i := v$ and \Send (input, $p_i$) $\rightarrow \mathcal{A}$

-- \OnInput \inmsg{\texttt{output}} from \Partyi:

	\qquad \If $p_i \in \mathcal{H}$ and $x_i = \bot$: ignore \Else:

		\qqquad $*$ \If $t_i > 1$:

			\qqqquad Set $t_i := t_i - 1$. \If (now) $t_j = 1$ for all $p_j \in \mathcal{H}$:

				\qqqqquad Set $\ell := \ell + 1$ and $t_j := |\mathcal{P}|$ for all $p_j \in mathcal{P}$. \Send (activated,$p_i$) $\rightarrow \mathcal{A}$

		\qqquad $*$ \Else If $t_i = 1$ and $\ell < Rnd$, \Send (early) $\rightarrow p_i$

		\qqquad $*$ \Else:

			\qqqquad -- \If $x_j \neq \bot$ and for all $p_i \in \mathcal{H}$, and $y_1,...,y_n$ are not set: $r \xleftarrow{\$} R$ 
			
				\qqqqquad set $(y_1,...,y_n) := f(x_1,...,x_n, r)$

			\qqqquad -- Output $y_i$ to $p_i$

\end{bbox}

	\label{fig:sfe}
	\caption{The ideal functionality for secure function eveluation from Katz~\cite{synchronousuc}. It's parameterized by the function $f$ and a polynomial $Rnd$ representing the upper bound on the number of rounds the real world protocol takes. It proceeds in rounds where eact party requires $|\mathcal{P}|$ activations. And it's WRONG.}
\end{figure}

Below is the updated version with the correct number of activations.

\begin{figure}
	\begin{bbox}[title={$\F_{\msf{sfe}}^{f,Rnd} (\mathcal{P})$}]

For each $p_i \in \mathcal{P}$ initialize $x_i = y_i := \bot$, delay $t_i := |\mathcal{P}|$.
Global $\ell := 1$

-- \OnInput \inmsg{\texttt{input}}{$v$} from \Partyi:

	\qquad Set $x_i := v$ and \Send (input, $p_i$) $\rightarrow \mathcal{A}$

-- \OnInput \inmsg{\texttt{output}} from \Partyi:

	\qquad \If $p_i \in \mathcal{H}$ and $x_i = \bot$: ignore \Else:

		\qqquad $*$ \If $t_i > 0$:

			\qqqquad Set $t_i := t_i - 1$. \If (now) $t_j = 0$ for all $p_j \in \mathcal{H}$ and $\ell < Rnd$:

				\qqqqquad Set $\ell := \ell + 1$ and $t_j := |\mathcal{P}|$ for all $p_j \in mathcal{P}$. \Send (activated,$p_i$) $\rightarrow \mathcal{A}$

		\qqquad $*$ \Else If $t_i = 0$ and $\ell < Rnd$, \Send (early) $\rightarrow p_i$

		\qqquad $*$ \Else:

			\qqqquad -- \If $x_j \neq \bot$ and for all $p_i \in \mathcal{H}$, and $y_1,...,y_n$ are not set: $r \xleftarrow{\$} R$ 
			
				\qqqqquad set $(y_1,...,y_n) := f(x_1,...,x_n, r)$

			\qqqquad -- Output $y_i$ to $p_i$

\end{bbox}

	\label{fig:sfe}
	\caption{Same ideal functionality as Figure~\ref{fig:sfe} but corrected}
\end{figure}

\begin{figure}
	\begin{bbox}[title={\textbf{Wrapper} $\mathcal{W}_{\msf{katz}} ( \pi )$}]

Initialize $\msf{todo} := \emptyset, t := 0, \msf{output} := \bot, \msf{ready} := 0, \msf{round} := 1$

\vspace{2mm} \hrule \vspace{2mm}

-- \OnInput \inmsg{input}{$x$} from $p_i$:  Forward \inmsg{input}{$x$} to $\pi$

-- \OnInput \inmsg{output} from $\mathcal{Z}$

	\qquad \If $t > 0$: Set $t = t-1$  If $\msf{todo} \neq \emptyset$:

		\qqquad Pop $(pid, msg) \leftarrow \msf{todo}$ and execute \Send $(msg) \rightarrow \Fbdsec$ where $j = pid$ and $i$ is the pid of $\pi$.

	\qquad \Else \If $t == 0$: Set $t = t-1$ and \Send $(\texttt{RoundOK}) \rightarrow \Fclock$

	\qquad \Else:
	
		\qqquad \If $\msf{ready}$: \Output \msf{output}

		\qqquad \Else \Output $(\texttt{early})$

\vspace{2mm} \hrule \vspace{2mm}

-- \OnInput \inmsg{Send}{pid}{msg} from $\pi$:

	\qquad Append $(pid, msg)$ to $\msf{todo}$

\end{bbox}

	\label{fig:wrapper}
\end{figure}


\subsection{Broadcast with Synchronous Wrapper}
This section descibres how the synchronous wrapper works.
The example is bracha broadcast with the functionalities, protocols, simulator and wrapper listed below.

\paragraph{Synchronous Wrapper}
The synchronous wrapper handles delaying execution of codeblocks within a upper bound $\Delta$ that is fixed by the calling protocol. 
The synchronous wrapper provides a simple interface to the parties and functionalities, defined in Figure~\ref{fig:wrapper:synchronous}.
The simplest example to demonstrate how the wrapper works is with a simple synchronous, point-to-point channel, $\F_{sync-chan}$ (Figure~\ref{fig:functionality:channel}).

In order to deliver an output, send a message, or execute an arbitrary code block with in a certain number of clock rounds, an ITM can ``schedule'' a piece of code to execute within a number of rounds.
In the definition of $\F_{sync-chan}$ in Fig~\ref{fig:functionality:channel}, a message needs to be delivered to the recipient in a synchrous manner so the functionality schdules the code block with the wrapper (for simplicty, we show two version of the same p2p channel: (a) one where the intuitive notion of synchronous code execution is captures and (b) where we show the actual messages passed from the functionality and the wrapper). 

\begin{figure}
	\begin{bbox}[title={$\F_{\msf{sync-chan}(p_s, p_r, r, \Delta)}$}]

\OnInput \inmsg{send}{msg} $d \token$ from $p_s$:
	\begin{renumerate}
	\item In $O(\Delta): ~\Send (msg, d \token)$
	\end{renumerate}
\end{bbox}

	\caption{A basic point-to-point channel functionality that delivers the message in a synchronous way.}
	\label{fig:functionality:channel}
\end{figure}

When a codeblock is schedule in the wrapper, it is immediately assigned the maximum possibe delay and added to the queue of schedule codeblocks.
The adversary is made aware of the round and the index into the queue for this new codeblock.
Additionally, a delay of 1 is added to the internal delay of the wrapper.
Once schedule, there are two ways for the code block to eventually be executed.
The first way is the adversary can decide to send the (\texttt{exec},$rnd$,$idx$) message to the wrapper to force the execution of a code block whenever it wants.
The second, method is through the environment polling the wrapper.
The purpose of allowing the environmen to force the wrapper to execute codeblocks is to prevent the adversary from having complete control over the wrapper and stop rounds from progressing.

\paragraph{Polling}
When the environment polls the wrapper, it decrements a delay counter (recall the delay is incremented when a new codeblock is schedule and can be further delayed by the adversary).
After sufficient polling, the delay reaches 0 in round $r$ and the next available code-block is popped from the queue and executed. 
The wrapper skips ahead to the first round $r' \geq r$ that contains a codeblock to execute.

\begin{figure}
	%\begin{bbox}[title={Wrapper $\mathcal{W} (\mathcal{F},\mathcal{C}_1,...,\mathcal{C}_k)$}]
%
%Initialize $\msf{outputs} := \emptyset$, $\msf{buffer} := \emptyset$
%
%\OnInput $\inmsg{buffer}{msg}{\delta}{P_i}$ from $\mathcal{F}$:
%	
%	\quad $\msf{buffer}[\Gledger.\msf{rnd}+\delta].\msf{append}(msg,P_i)$ 
%
%\OnInput \inmsg{read} from \Partyi:
%
%	\quad $\msf{out} := \msf{outputs}[P_i]$
%
%	\quad $\msf{outputs}[P_i] := \emptyset$
%
%	\quad $\Send \msf{out} \rightarrow P_i$
%
%\Allinputs m from \Partyi:
%
%	\quad \Send $m \rightarrow \F$
%
%\vspace{2mm} \hrule \vspace{2mm}
%
%When activated, do the following subroutine before processing the message:
%
%	\quad \For $(msg,P_i) \in \msf{buffer}[\Gledger.\msf{rnd}]$:
%
%		\qquad \Send $(\msf{deliver},msg,P_i) \rightarrow \F$
%
%		\qquad $\msf{outputs}[P_i].\msf{append}(msg)$
%
%\end{bbox}
\begin{bbox}[title={\textbf{Wrapper} $\mathcal{W}_{\msf{synchronous}} (\mathcal{F})$}]

Proceed in rounds starting in round $r=1$.

-- On first activation of $p_i$ in round $r$:

	\qquad \For $p_j \in \mathcal{P}$:

		\qquad \quad Get message $m$ from $\Fsync{p_j}{p_i}$

		\qquad \quad Deliver message to \F

	\todo{needs to be finalized in code first}

\end{bbox}

	\caption{This wrapper implements the enhancements made to the protocol to conform to the interace of the ideal functionality \Fbc. It reads in incoming messages in the first activation of a round and uses the next $|\mathcal{P}|$ activations to send messages to others.}
	\label{fig:wrapper:synchronous}
\end{figure}

\paragraph{Bracha}
The Bracha broacast protcol is shown in Figure~\ref{fig:prot_bracha_ours}. 
The protocol proceeds in a few rounds where VAL, ECHO, and READY messages exchanged until parties are ready to commit to a dealer input value $v$.

\begin{figure}
\begin{subfigure}{\textwidth}
	
\begin{bbox}[title={$\F_{\msf{bcast}} (\mathcal{D}, \mathcal{P}=p_1,...,p_n)$}]

\OnInput \inmsg{input}{v}, $n(4n+1) \token$ from $\mathcal{D}$:
	\begin{renumerate}
	\item For $p_i \in \mathcal{P}$:
		\begin{ritemize}
		\item In $O(4 \Delta)$: $\{\Send v \rightarrow p_i\}$
		\end{ritemize}
	\item \Leak (input, $v$), $n(4n+1) \token$ $\rightarrow \mathcal{A}$
	\end{renumerate}

\end{bbox}

	\label{fig:functionality:broadcast_import}
	\caption{A simple broadcast functionality that delivers the output to all the other parties within $O(\Delta)$ rounds. The functionality is described at a high level, with more specific messages and wrapper interaction shown below.}
\end{subfigure}
\newline
\begin{subfigure}{\textwidth}
	\begin{bbox}[title={$\F_{\msf{bcast}} (\mathcal{D}, \mathcal{P}=p_1,...,p_n)$}]

-- \OnInput \inmsg{input}{v}, $n(4n+1) \token$ from $\mathcal{D}$:

	\qquad For $p_i \in \mathcal{P}$:
		
		\qqquad \Send (schedule, $\{\Send v \rightarrow p_i\}$, $4 \Delta$) $\rightarrow \mathcal{W}_{sync}$

	\qquad \Send (leak, (input, $v$), $n(4n+1) \token$) $\rightarrow \mathcal{W}_{sync}$

\end{bbox}

	\label{fig:functionality:broadcast_import_real}
	\caption{This broacsat functionality illustrates the actual messages being passed between the functionality above and the wrapper. The wrapper handles synchronous delivery of messages {\em and} leaks.}
\end{subfigure}
\end{figure}

\begin{figure}
	\begin{bbox}[title={$\Pi_{\msf{Bracha}} (\mathcal{D}, \mathcal{P} = p_1,...,p_n)$ in $\F_{\msf{sync-chan}}$-hybrid}]

Initialize $\msf{BQ} := \frac{\msf{ceil}(n+t)}{2}$, $\msf{init} := crnd$, $\msf{out} := \emptyset$

\vspace{2mm} \hrule \vspace{2mm}

% dealer input INPUT
{\bf Dealer $\mathcal{D}$ Protocol}

\OnInput \inmsg{input}{m}{$ n(4n+1) \token $} from $\mathcal{Z}$:
	\begin{renumerate}
	\item \For $p_i \in \mathcal{P}$:

		\quad  \Send $\msf{VAL}(m),4n \token \rightarrow \Fsync{\mathcal{D}}{p_i}$
	\end{renumerate}

\vspace{2mm} \hrule \vspace{2mm}

{\bf Party $p_i$ Protocol}

% on input VAL
\OnInput \inmsg{$\msf{VAL}(m)$}{$4n \token$} from $\Fchan{\mathcal{D}}{p_i}$ (once):
	\begin{renumerate}
	\item \For $p_j \in \mathcal{P}$: 
	
	\quad \Send $\msf{ECHO}(m), 3 \token \rightarrow \Fchan{p_i}{p_j}$
	\end{renumerate}

\OnInput \inmsg{$\msf{ECHO}(m)$}{$3 \token$} from $\Fchan{p_j}{p_i}$:
	\begin{renumerate}
	\item \If received $\msf{ECHO}(m)$ from $\msf{BQ}$ parties:
		\begin{ritemize}
		\item \For $p_j \in \mathcal{P}$: 
		
		\quad \Send $\msf{READY}(m), 0 \token \rightarrow \Fchan{p_i}{p_j}$ \\
		\end{ritemize}
	\end{renumerate}
% on input READY

\OnInput \inmsg{$\msf{READY}(m)$}{$0 \token$} from $\Fchan{p_j}{p_i}$:
	\begin{renumerate}
	\item \If received $\msf{READY}(m)$ from $2t+1$ parties:
		\begin{ritemize}
		\item $\Send m \rightarrow \mathcal{Z}$
		\end{ritemize}
	\end{renumerate}

\vspace{2mm} \hrule \vspace{2mm}

\end{bbox}


	\label{fig:prot:bracha_ours}
\end{figure}

\begin{figure}
	\begin{bbox}[title={Simulator $\mathcal{S}_{bracha} (\mathcal{D}, \mathcal{P}, \Delta)$}]

Simulate real world parties $p_1',...,p_n'$ and the simulated dealer $\mathcal{D}'$.

Initialize $\msf{idealqueue} := \emptyset, idealdelay := 0$

\vspace{2mm} \hrule \vspace{2mm}

\OnInput \inmsg{\texttt{get-leaks}} from $\mathcal{Z}$:
	\begin{renumerate}
	\item $leaks \leftarrow$ \{ \Send (\texttt{get-leaks}) $\rightarrow \mathcal{W}_{sync}$\}

	\item For $l \in leaks$:
		\begin{renumerate}
		\item If $l$ is (input, v), $n(4n+1) \token$ from $\F_{bcast}$:

			\quad Simulate (input, $v$, $n(4n+1) \token$) $\rightarrow \mathcal{D}'$ 

		\item Else If $l$ is (schedule, $rnd$, $idx$) from $\F_{bcast}$:

			\quad Map $p_i$ to $(rnd,idx)$ for the $i$th such leak.

			\quad $idealdelay = idealdelay + 1$

		\item Else:

			\quad $idealdelay = idealdelay + 1$
		\end{renumerate}
	%\qquad $leaks \leftarrow$ \{ \Send (\texttt{get-leaks}) $\rightarrow \mathcal{W}_{sync}'$\}
	\item $leaks \leftarrow \msf{SimGetLeaks}$

	\item \Send $leaks \rightarrow \mathcal{Z}$
	\end{renumerate}

\OnInput \inmsg{poll} from $\mathcal{Z}$:
	\begin{renumerate}
	\item execute \msf{Poll}
	\end{renumerate}

\OnInput \inmsg{delay}{$d \token$} from $\mathcal{Z}$:
	\begin{renumerate}
	\item Simulate $(delay, d \token) \rightarrow \mathcal{W}_{sync}'$

	\item \Send $(\texttt{delay}, d \token) \rightarrow \mathcal{W}_{sync}$

	\item $idealdelay = idealdelay + d$
	\end{renumerate}

\OnInput \inmsg{exec}{$rnd$}{$idx$} from $\mathcal{W}_{sync}$:
	\begin{renumerate}
	\item Simulate $(\texttt{exec}, rnd, idx) \rightarrow \mathcal{W}_{sync}'$

	\item Forward any messages from a simulated part $p_i'$ or $\mathcal{A}'$
	\end{renumerate}

\end{bbox}

	\label{fig:sim:bracha_ours}
\end{figure}

\begin{figure}
	\begin{subfigure}{\textwidth}
	\begin{bbox}[title={Algorithm $\msf{SimGetLeaks}$}]

$leaks \leftarrow$ \{ \Send (\texttt{get-leaks}) $\rightarrow \mathcal{\mathcal{A}'}$ \}

$n \leftarrow \#$ of (schedule,...) messages in $leaks$

$idealdelay = idealdelay + n$

\{ \Send (delay, $n \token$) $\rightarrow \mathcal{W}_{sync}$ \}

\end{bbox}



	\label{fig:algo:simgetleaks}
	\caption{Function to get leaks from the simulated real world and update the delay of the ideal world if new codebloks have been schedules in the simulated wrapper, $\mathcal{W}_{sync}'$.}
	\end{subfigure}
	\newline
	\begin{subfigure}{\textwidth}
	\newcommand{\idealdelay}{{\color{Blue} \msf{idealdelay} }}

\begin{bbox}[title={Algorithm $\msf{Poll}$}]

\begin{renumerate}

  	\item $\idealdelay \minuseq 1$
  	
  	\item If $\idealdelay = 0$:
  	 
  		\quad \Send $(\texttt{delay}, 1 \token) \rightarrow \mathcal{W}_{sync}$

  		\quad $\idealdelay = 1$

  	\item \Send $(\texttt{poll},) \rightarrow \mathcal{W}_{sync}'$
 
  	\item If output $m$ from party $P_i'$:

			\quad Call $\msf{SimGetLeaks}$

			\quad Call \msf{SimPartyOutput}(m, $P_i'$)
		
		Else if $m$ from $\mathcal{A}'$:

			\quad Send $m \rightarrow \mathcal{Z}$

\end{renumerate}

\end{bbox}

	\label{fig:algo:poll}
	\caption{Function that forwards a (\texttt{poll}) message to the simulated wrapper and waits for a message back from a simulated party or the adversary. If an output is given from a party and the fnuctionality, $\F_{bcast}$ functionality did not leak an input, the dealer is corrupt. Therefore, the simulator gives input to $\F_{bcast}$ and delivers that party's output.}
	\end{subfigure}
\end{figure}


%\section{Ledger Functionality}
%\begin{figure}[h]
	
\begin{bbox}[title=\msf{ExecTx(to, val, data, from)}]

$\msf{nonces[from]} \lar \msf{nonces[from]} + 1$

\If $\msf{balances[from]} < \msf{val}$: \reject

$\msf{balances[from]} \lar \msf{balances[from]} - \msf{val}$

$\msf{balances[to]} \lar \msf{balances[to]} + \msf{val}$

$\msf{receipts[from,nonces[from]]} \lar \msf{CreateTxRef(val, from)}$

\If $\msf{to} \in \msf{contracts}$:

	\quad $ret \lar \msf{Exec(to, val, data, from)}$

	\quad $\msf{txs[from, nonces[from]]} \lar ret$

\end{bbox}

	\label{fig:algorithm:exetx}
\end{figure}

\begin{figure}[h]
    \begin{bbox}[title=\msf{ExecContractCreate(addr, val, data, from, private)}]

$\msf{nonces[from]} \lar \msf{nonces[from]} + 1$

\If $\msf{balances[from]} < \msf{val}$: \reject

$\msf{balances[from]} \lar \msf{balances[from]} - \msf{val}$

$\msf{balances[to]} \lar \msf{balances[to]} + \msf{val}$

$(\msf{functions}, \msf{args}) := \msf{data}$

$r \lar \msf{functions.init}(args)$

$\msf{contracts[addr]} = \msf{functions}$

$\msf{restricted[addr]} = \msf{private}$

\If $\neg r$:

\quad $\msf{balances[from]} \lar \msf{balances[from]} + \msf{val}$

\quad $\msf{balanaces[to]} \lar \msf{balances[to]} - \msf{val}$
\end{bbox}

    \label{fig:algorithm:execcreate}
\end{figure}

\begin{figure}[h]
    \begin{bbox}[title=$\globalf{G}_{\msf{ledger}}$]

Initialize $\msf{txqueue} := \edict$, $\msf{contracts} := \edict$, $\msf{newtxs} := \edict$, $\msf{nonces} := \edict$ \msf{balances} := \edict, $\Delta := 8$, $rnd := 0$\\

\OnInput \transfermsg from \Partyi:

	\quad If $\msf{balances[fro]} < \msf{val}$: {\bf reject}

	\quad $\msf{nonces[from]} \lar \msf{nonces[from]} + 1$

	\quad $\msf{newtxs[from,nonces[from]} \lar \transfermsg$
	
	\quad {\bf leak} \transfermsg to \Adv

\OnInput \createmsg from \Partyi:

	\quad If $\msf{balances[from]} < \msf{val}$: {\bf reject}
	
	\quad $\msf{nonces[from]} \lar \msf{nonces[from]} + 1$

	\quad $caddr \lar \msf{ComputeAddr}(from)$
	
	\quad If $caddr \neq addr$: \reject

	\quad If \msf{len(data)} = 0: \reject

	\quad $\msf{newtxs[from,nonces[from]} \lar \transfermsg$

	\quad {\bf leak} \createmsg to \Adv

\OnInput \inmsg{tick}{addr} from \Partyi:

	\quad $rnd += 1$

	%\quad $\msf{balances[sid,pid]} \pluseq 1000000$
	\quad $\msf{balances[addr]} \pluseq 1000000$

	\quad \For \msf{tx} \In \msf{txqueue[rnd]}: 

		\qquad \If $tx[0] = \msf{transfer}$:
			
			\qqquad $(\msf{transfer, to, val, data, from}) \lar tx$

			\qqquad \msf{ExecTx(to, val, data, from)}

		\qquad If $tx[0] = \msf{contractcreate}$:

			\qqquad $(\msf{contractcreate, addr, val, data, private, from}) \lar tx$
	
			\qqquad \msf{ExecContractCreate(addr, val, data, private, from)}

\hrulefill

\OnAdvInput \inmsg{delayTx}{from}{nonce}{rounds} from \Adv:

	\quad $tx \lar \msf{newtxs[from,nonce]}$

	\quad Add $tx$ to \msf{txqueue[rnd + rounds]}

	\quad Remove $tx$ from \msf{newtxs}

\OnAdvInput \inmsg{tick}{addr}{permutation} from \Adv:
	
	\quad Apply $permutation$ to \msf{txqueue[rnd]}

	\quad Run honest party mining with \msf{addr}

\end{bbox}

	\caption{Ideal functionality representing a basic ledger with adversarial methods for delaying/reordering transactions and smart contract support}
	\label{fig:functionality:ledger}
\end{figure}

\begin{figure}[h]
	\begin{bbox}[title=Protection Wrapper $\f{W}_p$]

\OnInput \inmsg{transfer}{to}{val}{data}{from} from \Partyi:

	\quad $to \lar$

\end{bbox}

	\caption{Protection wrapper for the ledger to maintain indistinguishability.}
	\label{fig:wrapper:protected}
\end{figure}


%
%\section{Payment Channel}
%\subsection{Unidirectional Payment Channels}

In the unidirectional channel there are two parties, $P_s$ and $P_r$, where only the ``sender'' $P_s$ is sending payments to $P_r$.
The incentive structure in this case is simpler than bi-directional payments because one of the two paties, the receiver, has an incentive to always close with the most recent state of the channel.
Therefore, we consider the receiver's request to close the channel a ``cooperative close''. 
When the sender wants to close the channel, it can tell the receiver and submit a state signed by both of them---also a ``cooperative close''.
Otherwise, it submits a balance signed by only itself and initiates an ``uncooperative close''.

Assumptions:
\begin{itemize}
  \item The channels are already open and the protocol is parameterized by the initial balances of the two parties
  \item ``on chain'' operations like close and settle take O(delta) rounds through a smart contract and atomic broadcast
  \item ``off chain'' communication is instant, i.e. O(1) rounds to send message
  \item contract outputs are broadcast immediately
  \item ``cooperative close'' is only one on-chain operation, so takes O(delta)
  \item ``uncooperative close'' is (1. sender submits state and starts disute, 2. receiver can counter with later state and channel is settled): O(2*delta)
\end{itemize}


\begin{figure}[!htb]
	\begin{bbox}[title={$\F_{\msf{abc}} (C, P_s, P_r, \Delta)$}]

Initialize $buf = \{\}$

\vspace{2mm} \hrule \vspace{2mm}

\OnInput \inmsg{bcast}{msg} from $P_i$:
	\begin{renumerate}
		\item Append $(msg, P_i)$ to $buf$
		\item For $p_i$ in $P_s,P_r$:
		\begin{renumerate}
			\item \msf{codeblock} = \{
			
				\quad Send $buf \rightarrow p_i$

			\}
			\item Send (schedule, \msf{codeblock}, $\Delta$) $\rightarrow \mathcal{W}_{sync}$
		\end{renumerate}
	\end{renumerate}
\end{bbox}

\begin{bbox}[title={$\mathcal{C}_{pay}(P_s,P_r, \msf{balances}, \Delta)$}]

Intialize $T_{settlement} := 2 \Delta$, $T_{deadline} := 0$, $\msf{nonce} := 0$,


$\msf{state} := (\msf{balances}[P_s], \msf{balances}[P_r], \msf{nonce})$, $\msf{FLAG} := \msf{OFFCHAIN}$

\vspace{2mm} \hrule \vspace{2mm}

\OnInput \inmsg{close}{\msf{state'}}{$\msf{sig}_s$}{$\msf{sig}_r$} from $P_i$:
	\begin{renumerate}
		\item Assert $\msf{flag} = \msf{OFFCHAIN}$
		\item Assert $\msf{CheckSig}(P_s,\msf{sig}_s, \msf{state}')$
		
		\item $(b_s, b_r, n) \leftarrow \msf{state}'$

		\item Set $\msf{nonce} = n$, $\msf{state} = \msf{state}'$

		If $\msf{CheckSig}(P_r, \msf{sig}_r, \msf{state}')$:
		\begin{renumerate}
			\item Set $\msf{flag} = \msf{Closed}$
			\item Send (\msf{Closed}, \msf{state}') $\rightarrow \F_{\msf{abc}}$ 
		\end{renumerate}

		Else:
		\begin{renumerate}
			\item Set $\msf{flag} = \msf{UnCoopClose}$ 
			\item Set $T_{deadline} = T_{now} + T_{settleent}$
			\item Send (\msf{UnCoopClose}, \msf{state}', $T_{deadline}$) $\rightarrow \F_{\msf{abc}}$
		\end{renumerate}
		
	\end{renumerate}

\OnInput \inmsg{challenge}{\msf{state}'}{$\msf{sig}_s$}{$\msf{sig}_r$} from $P_i$:
	\begin{renumerate}
		\item Assert $\msf{flag} = \msf{UnCoopClose}$

		Assert $\msf{CheckSig}(P_s, \msf{sig}_s, \msf{state}')$

		Assert $\msf{CheckSig}(P_r, \msf{sig}_r, \msf{state}')$
		\item $(b_s, b_r, n) \leftarrow \msf{state}'$
		\item Assert $n > \msf{nonce}$

		\item Set $\msf{flag} = \msf{Closed}$
		
		Set $\msf{state} = \msf{state}'$
		\item Send (\msf{Closed}, \msf{state}') $\rightarrow \F_{\msf{abc}}$
	\end{renumerate}
\end{bbox}

\begin{bbox}[title={$\F_{\msf{off-chain-chan}}(P_s, P_r)$}]

\OnInput \inmsg{send}{msg} from $P_s$:
	\begin{renumerate}
			\item \msf{codeblock} = \{
			
				\quad Send msg $\rightarrow P_r$

			\}
			\item Send (schedule, \msf{codeblock}, 1) $\rightarrow \mathcal{W}_{sync}$
	\end{renumerate}

\end{bbox}

	\caption{Permissioned atomic broadcast parameterized by contract code $C$.}
\end{figure}



%\begin{figure}[h]
%	\begin{bbox}[title=$U_{pay}$]

$U_{pay} (\msf{state}, (\msf{input_L},\msf{input_R}), \msf{aux}_{in})$:

\quad \If $\msf{state} = \bot$: $\msf{state} := (0,\emptyset,0,\emptyset)$

\quad parse \msf{state} as $(\msf{cred_L},\msf{oldarr_L},\msf{cred_R},\msf{oldarr_R})$

\quad parse $\msf{aux}_{in}$ as $\{ \msf{deposits}_i \}_{i \in \{L,R\}}$

\quad \For $i \in \{L,R\}$:

	\qquad \If $\msf{input}_i = \bot$: $\msf{input}_i := (\emptyset,0)$

	\qquad parse $\msf{input}_i$ as $\msf{arr}_i,\msf{wd}_i$

	\qquad $\msf{pay}_i := 0, \msf{newarr}_i := \emptyset$

	\qquad \While $\msf{arr}_i \neq \emptyset$:

		\qqquad $e \leftarrow \msf{pop}(\msf{arr}_i)$

		\qqquad \If $e + \msf{pay}_i \leq \msf{deposits}_i + \msf{cred}_i$:

			\qqqquad $\msf{newarr}_{\neg i} \leftarrow e$

			\qqquad $\msf{pay}_i += e$

	\qquad \If $\msf{wd}_i > \msf{deposits}_i + \msf{cred}_i - \msf{pay}_i: \msf{wd}_i := 0$

\quad $\msf{cred_L} += \msf{pay_R} - \msf{pay_L} - \msf{wd_L}$

\quad $\msf{cred_R} += \msf{pay_L} - \msf{pay_R} - \msf{wd_R}$

\quad \If $\msf{wd_L} \neq 0$ or $\msf{wd_R} \neq 0$:

	\qquad $\msf{aux}_{out} := (\msf{wd_L},\msf{wd_R})$

\quad \Else: $\msf{aux}_{out} := \bot$

\quad $\msf{state} := (\msf{cred_L},\msf{newarr_L},\msf{cred_R},\msf{newarr_R})$

\quad \Return $(\msf{aux}_{out}, \msf{state})$

\end{bbox}

%	\caption{Update function for a payment channel. Given as a parameter to \Fstate. It defines the format of the \msf{state} and its updates.}
%\end{figure}
%
%\begin{figure}[h]
%	\begin{bbox}[title=$\Pi_{pay}$: $\msf{Contract_{pay}}$]

\Init $(\msf{P_L}, \msf{P_R})$:

\quad $\msf{deposits}_L, \msf{deposits}_R := 0$

% deposit($X)
\OnInput \inmsg{deposit}(tx) from \Partyi:

	\quad $\msf{deposits}_i += \msf{tx.value}$

	\quad $\msf{out}(\msf{deposits}_L, \msf{deposits}_R)$

% aux_out 
\OnInput \inmsg{output}{\msf{aux_{out}}}{\msf{tx}}:

	\quad parse $\msf{aux_{out}}$ as $(\msf{wd_L},\msf{wd_R})$

	\quad \For $i \in \{L,R\}$: $\msf{send}(P_i, \msf{wd_i})$

\end{bbox}

%	\caption{Contract pay}
%\end{figure}
%
%\begin{figure}[h]
%	\begin{bbox}[title=$\Pi_{pay}$]

Initialize $\msf{arr_i} = \emptyset, \msf{pay_i} = 0, \msf{wd_i} = 0, \msf{paid_i} = 0$

$\msf{Contract_{pay}}$ identifier $\mathcal{C}$ 

\Send $(\emptyset, 0) \rightarrow \F_{state}$

% New state from F_state
\OnInput $(\msf{cred_L}, \msf{new_L}, \msf{cred_R}, \msf{new_R})$ from $\Fstate$:

	\quad \For $e \in \msf{new_i}$:

		\qquad \textbf{Output} $(\msf{receive}, e)$

		\qquad $\msf{paid_i} += e$
	
	\quad \Send $(\msf{arr_i}, \msf{wd}-\msf{wdn}) \rightarrow \Fstate$

	\quad $\msf{arr_i} \leftarrow \emptyset$

	\quad $\msf{wdn_i} \leftarrow \msf{wd_i}$

% Pay($X)
\OnInput \inmsg{pay}{\$X} from \Env:

	\quad $\msf{Contract_{Pay}} \leftarrow \Gledger.\msf{contract}(\mathcal{C})$

	\quad \If $\$X \leq \msf{Contract_{Pay}}.\msf{deposits_i} + \msf{paid_i} - \msf{pay_i} - \msf{wd_i}$:

		\qquad $\msf{arr_i} \leftarrow \$X$

		\qquad $\msf{pay_i} += \$X$

% Withdraw($X)
\OnInput \inmsg{withdraw}{\$X} from \Env:

	\quad $\msf{Contract_{Pay}} \leftarrow \Gledger.\msf{contract}(\C)$

	\quad \If $\$X \leq \con{Pay}.\msf{deposits_i} + \msf{paid_i} - \msf{pay_i} - \msf{wd_i}$:

		\qquad $\sf{wd_i} += \$X$

\end{bbox}

%	\caption{Local protocol for parties to follow for a payment channel between two parties. Parties can pay, deposit into, or withdraw from the channel.}
%\end{figure}
%
%\begin{figure}[h]
%	\begin{bbox}[title={$\Fstate (U, \mathcal{C}, \mathcal{P} = \{P_1,...,P_N\}, \Delta)$}]

Initialize $\msf{aux}_{in} := [\bot]$, $\msf{ptr} := 0$, $\msf{state} := \emptyset$, $\msf{buf} := \emptyset, \msf{rnd} := 0$

% environment can ping the functionality to check for contract outputs
\OnInput \inmsg{ping} from \Partyi:

	\quad $\msf{aux}_{in} := \Gledger.\msf{coutput}(\mathcal{C})$

	\quad append $\msf{aux}_{in}$ to $\msf{buf}$

	\quad $j := |\msf{buf}| - 1$

	\quad $\msf{ptr} := \msf{max}(\msf{ptr},j)$ 

\vspace{2mm} \hrule \vspace{2mm}

Proceed in rounds starting at $\msf{rnd} := 0$:

	$v_{\msf{rnd},i} := \bot, \forall i \in \mathcal{P}$

	% input from the party for tihs round
\OnInput \inmsg{m} from \Partyi:

	\quad \If $v_{\msf{rnd},i} = \bot$:

		\qquad $v_{\msf{rnd},i} := m$

		\qquad \Leak $(i,v_{\msf{rnd},i}) \rightarrow \Adv$

% step function to check conditions and execute state update
\OnInput \inmsg{\msf{step}} from \Partyi:

	% if round can be progressed (deadline has passed or all parties have inpuit)
	\quad \If $\left( \forall v_{\msf{rnd},i} :  v_{\msf{rnd},i} \neq \bot \right) \vee \left( \exists v_{\msf{rnd},i} : v_{\msf{rnd},i} \neq \bot \wedge \Gledger.\msf{rnd} > \msf{deadline} \right)$:

		% compute state update
		\qquad $(\msf{state},o) := U(\msf{state}, \{v_{\msf{rnd},i}\}_{i\in\mathcal{P}}, \msf{aux}_{in}[\msf{ptr}])$
		
		\qquad $\msf{rnd} := \msf{rnd} + 1$, $\msf{deadline} := \Gledger.\msf{rnd} + \Delta$	

		\qquad \If $\left( \forall P_i : P_i.\msf{ishonest} \right)$:
			
			\qqquad $\forall P_i : \Buffer (\msf{state}, 1, P_i)$

		\qquad \Else: $\forall P_i : \Buffer (\msf{state}, O(\Delta), P_i)$

		\qquad \If $o \neq \bot$:

		% transfer money for contract output
		\qqquad \Send $(\msf{transfer},\mathcal{C},0,(output, o), \bot) \rightarrow \Gledger$ 
		

\end{bbox}

%	\caption{The ideal functionality \Fstate. The functionality proceeds in rounds and waits for parties to provide input. When all parties have provided input or the round deadline has passed, a state update is executed. Contract output is given to \Gledger in the form of a transaction. Parties must explicitly \msf{ping} the functionality in order to make progress. }
%\end{figure}
%
%\begin{figure}[h]
%	\begin{bbox}[title={$\F_{\msf{pay}} (P_s, P_r, \msf{balances}, \Delta)$}]

Initialize $\msf{flag} := \msf{OPEN}$

\vspace{2mm} \hrule \vspace{2mm}

\OnInput \inmsg{pay}{v} from $P_s$:
	\begin{renumerate}
			\item In $O(1)$ rounds:  
				
				\quad \Require $\msf{flag} = \msf{OPEN}$
				
				\quad If $\msf{balances}[P_s] >= v$:
			
				\qquad $\msf{balances}[P_s] -= v$
				
				\qquad $\msf{balances}[P_r] += v$
				
				\qquad Send $(\msf{pay}, v) \rightarrow P_r$

		\item Leak $(\msf{pay}, v)$

	\end{renumerate}

\OnInput \inmsg{close} from $P_i$:
	\begin{renumerate}	
		\item If $P_i = P_r$ or ($P_i = P_s$ and $P_s$ is honest):
			\begin{renumerate}
			\item In $O(\Delta)$ rounds:  

			\qquad \Require $\msf{flag} = \msf{OPEN}$

			\qquad In $O(1)$ rounds:  Send $(\msf{close}, b_r, b_s) \rightarrow P_s$

			\qquad In $O(1)$ rounds: Send $(\msf{close}, b_r, b_s) \rightarrow P_r$

			\end{renumerate}

		Else:
		\begin{renumerate}
			\item In $O(k \times \Delta)$ rounds:

			\qquad \Require $\msf{flag} = \msf{OPEN}$

			\qquad In $O(1)$ rounds:  Send $(\msf{close}, b_r, b_s) \rightarrow P_s$

			\qquad In $O(1)$ rounds:  Send $(\msf{close}, b_r, b_s) \rightarrow P_r$

		\end{renumerate}
		
		\item Leak $(\msf{close}, b_r, b_s, P_i)$
		
	\end{renumerate}

\end{bbox}

%	\caption{The payment channel functionality. Unlike $\Fstate$, doesn't need any notion of rounds until it must deal with on-chain transactions for deposits. Buffering for $O(\Delta)$ rounds implies the adversary can choose the number.}
%\end{figure}



%\begin{figure}[h]
%	%\begin{bbox}[title={Wrapper $\mathcal{W} (\mathcal{F},\mathcal{C}_1,...,\mathcal{C}_k)$}]
%
%Initialize $\msf{outputs} := \emptyset$, $\msf{buffer} := \emptyset$
%
%\OnInput $\inmsg{buffer}{msg}{\delta}{P_i}$ from $\mathcal{F}$:
%	
%	\quad $\msf{buffer}[\Gledger.\msf{rnd}+\delta].\msf{append}(msg,P_i)$ 
%
%\OnInput \inmsg{read} from \Partyi:
%
%	\quad $\msf{out} := \msf{outputs}[P_i]$
%
%	\quad $\msf{outputs}[P_i] := \emptyset$
%
%	\quad $\Send \msf{out} \rightarrow P_i$
%
%\Allinputs m from \Partyi:
%
%	\quad \Send $m \rightarrow \F$
%
%\vspace{2mm} \hrule \vspace{2mm}
%
%When activated, do the following subroutine before processing the message:
%
%	\quad \For $(msg,P_i) \in \msf{buffer}[\Gledger.\msf{rnd}]$:
%
%		\qquad \Send $(\msf{deliver},msg,P_i) \rightarrow \F$
%
%		\qquad $\msf{outputs}[P_i].\msf{append}(msg)$
%
%\end{bbox}
\begin{bbox}[title={\textbf{Wrapper} $\mathcal{W}_{\msf{synchronous}} (\mathcal{F})$}]

Proceed in rounds starting in round $r=1$.

-- On first activation of $p_i$ in round $r$:

	\qquad \For $p_j \in \mathcal{P}$:

		\qquad \quad Get message $m$ from $\Fsync{p_j}{p_i}$

		\qquad \quad Deliver message to \F

	\todo{needs to be finalized in code first}

\end{bbox}

%	\caption{The wrapper $\mathcal{W}$ that provides common function for all functionalities. In $\Fstate$ for example, the wrapper enables functionalities to buffer sending output to the parties in the protocol. When the wrapper sends a message to its functionality $\F$, it does not constitute an \msf{ITM} to \msf{ITM} write as they are both running on the same \msf{ITM}.}
%\end{figure}

%\begin{figure}
%	\begin{bbox}[title={$\F_{\msf{bcast}} (p_L, p_1...p_n)$}]

Initialie $\msf{buffer} := \emptyset$, $\msf{lastRound} \leftarrow -1$, $\msf{round} \leftarrow 0$

\OnInput \inmsg{broadcast}{msg} from \Partyi:

	\quad \If $\Partyi \neq p_L$: ignore

	\quad \Leak $(\msf{msg},\msf{round} + 1) \rightarrow \mathcal{A}$

	\quad $\msf{buffer}[\msf{round}+1] \leftarrow \msf{msg}$


\OnInput \inmsg{deliver}{msg}{to} from $\mathcal{A}$:
	
	\quad \If $\msf{to} \notin (p_1,...,p_n)$: ignore

	\quad $m,r \leftarrow \msf{msg}$

	\quad \If $\msf{m} \in \msf{buffer}[r]$:

	\quad \quad \Send $(m) \rightarrow \msf{to}$

\vspace{2mm} \hrule \vspace{2mm}

When activated do that following first:

	\quad \Send $(\msf{clockread},) \rightarrow \globalf{G}_{\msf{clock}}$

	\quad $\msf{rnd} \leftarrow wait(\globalf{G}_{\msf{clock}})$

	\quad \If $\msf{rnd} > \msf{round}$:

	\quad \quad $\msf{lastRound} \leftarrow \msf{round}$
	
	\quad \quad $\msf{round} \leftarrow \msf{rnd}$

\end{bbox}

%\end{figure}
%
%\begin{figure}
%	\begin{bbox}[title={$\globalf{G}_{\msf{clock}}$}]

Intialize $\msf{registry} := \emptyset$, $\msf{dp} := \emptyset$, $\msf{sessionT} := \emptyset$

\OnInput \inmsg{register} from \Partyi:

	\quad \If $\msf{pid} \notin \msf{registry}[\msf{sid}]$:

	\quad \quad Add $\msf{pid}$ to $\msf{registry}[\msf{sid}]$

	\quad \quad \If $\msf{sid} \notin \msf{sessionT}$:
		
	\quad \quad \quad $\msf{sessionT}[\msf{sid}] := 0$

\OnInput \inmsg{clockread} from \Partyi:

	\quad \If $\msf{sid} \notin \msf{registry}$: ignore

	\quad \Send $\msf{sessionT}[\msf{sid}] \rightarrow P_i$ 

\OnInput \inmsg{clockupdate} from \Partyi:

	\quad \If $\msf{sid} \notin \msf{registry}$: ignore

	\quad $\msf{dp}[\msf{sid},\msf{pid}] := 1$

	\quad \If $\forall \msf{p}$, $\msf{dp}[\msf{sid},\msf{p}] = 1$:

	\quad \quad $\msf{sessionT}[\msf{sid}] += 1$

\end{bbox}

%\end{figure}
%
%\begin{figure}
%	\begin{bbox}[title={$\Pi_{\msf{state}} (\msf{sid}, \msf{pid}, U, \mathcal{C}_{\msf{aux}}, \mathcal{C}_{\msf{state}}, \msf{leader}, \msf{peers}=p_1,...p_n)$}]

Initialize $\msf{round} := 0$, $\msf{pinputs} := \emptyset$, $\msf{aux_in} = []$, $\msf{flag} := \msf{OK} \in \{\msf{OK},\msf{PENDING}\}$, $\msf{aux\_out} := \emptyset$, $\msf{state} := \emptyset, \msf{psigs} := \emptyset, \msf{lastRound} := -1$

$\msf{step} := input \in \{input,batch,commit\}$

\vspace{2mm} \hrule \vspace{2mm}

If $\msf{pid} = \msf{leader}$, do the following:

\OnInput \inmsg{INPUT}{$v_i$}{r} from \Partyi:

	\quad \If $r \neq \msf{round}$: \ ignore

	\quad \If first input from $P_i$ in round $r$: \ Add $v_i$ to $\msf{pinputs}$ 

	\quad \If $\forall p_i, v_i \in \msf{pinputs}$:
	
	\quad \quad \Send $(\msf{BATCH}, \msf{r}, \msf{aux_in}, \msf{pinputs}) \rightarrow \F_{\msf{bcast}}$ 


\OnInput \inmsg{SIGN}{$\sigma$}{r} from \Partyi:

	\quad \If $\msf{step} \neq commit$ or $\msf{r} \neq \msf{round}$ or $\msf{Verify}(\sigma, \msf{r}, \msf{aux\_out}, \msf{state}) \neq 1$: \ ignore

	\quad \If first sign from $P_i$ in round $r$: \ Add $(P_i,\msf{r},\sigma)$ to $\msf{psigs}$

	\quad \If $\forall p_i, (p_i,\msf{r},\_) \in \msf{psigs}$:

	\quad \quad \Send $(\msf{COMMIT}, \msf{r}, \{\sigma\}_{i}) \rightarrow \F_{\msf{bcast}}$


\vspace{2mm} \hrule \vspace{2mm}

{\bf If $\msf{flag} = \msf{OK}$}:

\OnInput \inmsg{input}{v} from $\mathcal{Z}$:

	\quad \If $\msf{step} \neq input$ or $r \neq \msf{round}$: \ ignore

	\quad $\msf{step} := batch$ 

	\quad \Send $(\msf{INPUTS}, \msf{v}, \msf{round}) \rightarrow \msf{leader}$ 

\OnInput \inmsg{\msf{BATCH}}{r}{aux\_in}{pinputs} from $\F_{\msf{bcast}}$:

	\quad \If $\msf{step} \neq batch$ or $r \neq \msf{round}$: \ ignore

	\quad $\msf{step} := commit$

	\quad todo: how to imply ``recent'' value of aux\_in??

	\quad $\msf{state},\msf{aux\_out} := U(\msf{state}, \msf{pinputs}, \msf{aux\_in}, \msf{round})$

	\quad $\sigma \leftarrow \msf{Sign}(r || \msf{aux\_out} || \msf{state})$

	\quad \Send $(\msf{SIGN}, \sigma) \rightarrow \msf{leader}$


\OnInput \inmsg{\msf{COMMIT}}{r}{$\{\sigma_r\}_i$} from $\F_{\msf{bcast}}$:

	\quad \If $\msf{r} \neq \msf{round}$ or $\msf{step} \neq commit$ or $\left( \bigvee_{\sigma_i} \msf{Verify}(\sigma_i,\msf{r},\msf{aux\_out},\msf{state}) = 0 \right)$: \ ignore

	\quad \msf{lastCommit} := $(\msf{state},\msf{aux_out},\{\sigma_r\}_i)$

	\quad \msf{lastRound} := \msf{r}

	\quad \msf{round} := \msf{lastRound}+1

	\quad $\msf{step} := input$

\end{bbox}

%\end{figure}
%
%\subsection{Extra}
%
%\begin{figure}
%	\begin{bbox}[title={\textbf{Functionality} $\F_{\msf{clock}} (\mathcal{P})$}]

Intialize $\forall p_i \in \mathcal{P}: d_i := 0$

\vspace{2mm} \hrule \vspace{2mm}

-- \OnInput \inmsg{RoundOK} from \Partyi:

	\dquad $d_i = 1$

	\dquad \If $\forall p_i \in \mathcal{P}: d_i = 1$:

	\dquad \quad $d_i := 0$ for all $p_i \in \mathcal{P}$

	\dquad \Leak $(\msf{switch},\mathbf{P_i}) \rightarrow \mathcal{A}$

-- \OnInput \inmsg{RequestRound} from \Partyi: 

	\dquad \Send $d_i \rightarrow \mathbf{P_i}$

-- \OnInput \inmsg{corrupt}{$p_i$} from $\mathcal{A}$:

	\dquad Set $\mathcal{H} := \mathcal{H} \cup \{p_i\}$

\end{bbox}


%\end{figure}
%
%\begin{figure}
%	\begin{bbox}[title={Wrapper $\mathcal{W}_{\msf{Eventually}} (\F)$}]

Initialize $\msf{crnd} := 0$, $\msf{lastcrnd} := -1$, $\msf{runqueue} := []$

\vspace{2mm} \hrule \vspace{2mm}

\OnInput \inmsg{eventually}{codeblock e} from $\F$

	\quad Add $e$ to $\msf{runqueue}$

	\quad \Leak $e \rightarrow \mathcal{A}$

\OnInput \inmsg{deliver}{idx} from $\mathcal{A}$:

	\quad $e \leftarrow \msf{runqueue}[idx]$

	\quad Delete $\msf{runqueue}[idx]$

	\quad {\bf Execute} $e$

\vspace{2mm} \hrule \vspace{2mm}

On every activation:

	\quad $\msf{rnd} \leftarrow \F_{\msf{clock}}.\msf{clockread}$

	\quad \If $\msf{rnd} \neq \msf{crnd}$:

		\quad \quad $\msf{lastcrnd} \leftarrow \msf{crnd}$

		\quad \quad $\msf{crnd} \leftarrow \msf{rnd}$

\end{bbox}

\begin{bbox}[title={$\F_{\msf{3PC}} (\mathcal{D}, \mathcal{P} = p_1,...,p_n, V_C)$}]

Initialize $\msf{buffer} := \emptyset$, $\msf{pending} := False$ %$\msf{flag} := \msf{OK} \in \{\msf{OK},\msf{PENDING}\}$

$quorum := 0$, $d_t := -1$

\vspace{2mm} \hrule \vspace{2mm}

\OnInput \inmsg{input}{T} from $\mathcal{D}$:

	\quad \If $\msf{pending}$: \reject

	\quad $\msf{pending} = True$

	\quad $d_t = \msf{crnd} + 2$

	\quad \For $p_i \in \mathcal{P}$:

		\quad \quad \msf{Eventually} \Send $\msf{ready} \rightarrow p_i$

%\OnInput \inmsg{send}{T} from $\mathcal{D}$:
%
%	\quad \If $\msf{flag} = \msf{PENDING}$: \reject
%
%	\quad $d_t = \msf{crnd} + 2$
%
%	\quad \For $p_i \in \mathcal{P}$:
%
%		\quad \quad \msf{Eventually} \Send $T \rightarrow p_i$
%

\OnInput \inmsg{status}{s} from \Partyi:

	\quad \If not $\msf{pending}$: \ignore

	\quad \If first ``\msf{status}`` by ${\bf P_i}$:

		\quad \quad \If $s = OK$: $ok = ok + 1$

		\quad \quad \If $s = Abort$: $abort = abort + 1$

	\quad \If $ok \geq V_C$:

		\quad \quad \msf{pending} = $False$, $ok,abort = 0$, $d_t = -1$

		\quad \quad \For $p_i \in \mathcal{P}$:
			
			\quad \quad \quad \msf{Eventually} \Send $\msf{commit}{T} \rightarrow p_i$

	\quad \If $abort \geq V_A$:

		\quad \quad \msf{pending} = $False$, $ok,abort = 0$, $d_t = -1$


%\OnInput \inmsg{commit}{T} from \Partyi:
%
%	\quad \If $T \neq \msf{buffer}[-1]$ or $\msf{flag} = \msf{OK}$: \reject
%
%	\quad \If first ``\msf{commit}'' on $T$ by ${\bf P_i}$:
%
%		\quad \quad $quorum = quorum + 1$
%
%	\quad \If $quorum \geq V_C$:
%
%		\quad \quad \msf{flag} = \msf{OK}
%
%		\quad \quad $quroum = 0$, $d_t = -1$
%
%		\quad \quad \For $p_i \in \mathcal{P}$:
%
%			\quad \quad \quad \msf{Eventually} \Send $\msf{commit}(T) \rightarrow p_i$
%

\vspace{2mm} \hrule \vspace{2mm}

On every activation:

	\quad \If \msf{pending} and $\msf{crnd} \geq d_t$:

		\quad \quad Remove last element in \msf{buffer}

		\quad \quad $d_t = -1$, $ok = 0$, $abort = 0$

\end{bbox}

%\end{figure}
%
%\newpage
%
%\begin{figure}
%	\begin{bbox}[title={\textbf{Functionality} $\F_{\msf{RPS}}$}]

-- \OnInput \inmsg{x} from $p_i$:

	\qquad Store $x_i$ for $p_i$

	\qquad \Send (ready) $\rightarrow \mathcal{A}$

-- \OnInput \inmsg{ok}{$p_i$} from $\mathcal{A}$:

	\qquad \If $x_1,x_2$ are set:

		\qqquad $w \leftarrow \msf{rps_comp}(x_1,x_2)$

		\qqquad  \Send $w \rightarrow p_i$

\end{bbox}

%\end{figure}
%
%\begin{figure}
%	\begin{bbox}[title={\textbf{Functionality} $\F_{\msf{RPS}}$}]

-- \OnInput \inmsg{x} from $p_i$:

	\qquad Store $x_i$ for $p_i$

	\qquad \Send (ready) $\rightarrow p_{\neg i}$

-- \OnInput \inmsg{ok}{$p_i$} from $p_i$:

	\qquad \If $x_1,x_2$ are set:

		\qqquad $w \leftarrow \msf{rps_comp}(x_1,x_2)$

		\qqquad  \Send $w \rightarrow p_{\neg i}$

\end{bbox}

%\end{figure}


%\begin{bbox}[title=asd]
%hello
%\end{bbox}


\bibliographystyle{plain}
\emergencystretch 1.5em
\bibliography{bibuccontracts}


\end{document}
