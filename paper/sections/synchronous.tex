Like the asynchronous communcation model in Section~\ref{sec:async} our synchronous communication model relies on a wrapper $\mathcal{W}_{\msf{sync}}$ which enforces execution of scheduled codeblocks within a maximum number of rounds.
x


%\subsection{Broadcast in Katz et al.}
%
%In this section we describe the ideal functionality and protocol for a simple byzantine broadcast in the synchronous model.
%The function is modelled after the secure function evaluation functionality, $\F_{\msf{SFE}}^{f,Rnd}$, described in ~\cite{katz-clock}.
%Secure function evaluation computes the output of a single function from the inputs of the parties.
%The function being evaluated must be computable in one round in the ideal world, meaning parties don't have to give input more than once.
%In our case, the ideal function for a byantine broadcast, $f$ (shown below), with a dealier, $\mathcal{D}$, just selects the dealer's input.
%
%\[ f_{bc}(x_1,...,x_n) := x_{\mathcal{D}} \]
%
%The functionality proceeds in rounds, $l$, until $Rnd$ rounds have elapsed.
%At the end of $Rnd$ rounds it computes the function $f$ on the inputs of all parties that have provided input and sends output to each of them.
%In every round, the functionality requires the environment to activate each party at least $|\mathcal{P}|$ times before increenting the round $l$. 
%Everytime a party is activated by a party asking for \msf{output}, the functionality activates the the simulator.
%With $n$ activations, the simulator can sufficiently simulate the real world protocol and use the activations provdided by the functionality, to activate and perform computation.
%For example, for a party $p_i$, on activation $k$ within a round, the simulator simulates $p_i$'s interaction with party $p_k$ in the real world.
%It will read messages from $p_k$ to $p_i$, perform any local computation, and potentially send messages to $p_k$ for subsequent rounds.
%
%Similarly, such a structure design must also be present in the real world, where the environment will provide the same $n$ \msf{output} to each itm.

%\begin{figure}[!h]
%	\begin{bbox}[title={$\Fbc (\mathcal{D}, \mathcal{P} = p_1,...,p_n)$}]

Intialize $x_\dealer := \bot, \ell := 1, \forall p_i : t_i = |\mathcal{P}|$

\vspace{2mm} \hrule \vspace{2mm}

-- \OnInput \inmsg{input}{$v$} from \Partyi:
	
	\qquad Set $x_\dealer := v$

	\qquad \Leak $v$ to $\mathcal{A}$

	\qquad Within $O(Rnd)$ rounds, deliver $f(x_1,...,x_n)$ to $\forall p_i$

%-- \OnInput \inmsg{output} from \Partyi:
%
%	
%	\qquad \If ($p_i = \dealer$) and ($x_\dealer$ not set): ignore 
%
%	\qquad \Else \If $(t_i > 0)$: Set $t_i := t_i - 1$
%
%		\qquad \quad \If $(\forall p_i \in \mathcal{H})$: Set $\ell := \ell + 1$
%
%	\qquad \Else \If $(t_i = 0)$ and $(\ell < Rnd)$: \Send (early) $\rightarrow p_i$
%
%	\qquad \Else \If $(y_1,...,y_n)$ not set:
%
%		\qquad \quad Set $y_1,...,y_n := x_\dealer$
%
\end{bbox}


%	\label{fig:functionality:broadcast_high}
%	\caption{The ideal functionality for the synchronous Byzantine broadcast protocol from~\cite{bracha}. This functionality abstracts away UC-related details for clarity. This description of the functionality will actually be compiled to the functionality below which is how the program be written to adhere to the UC framework.}
%\end{figure}
%
%\begin{figure}[!h]
%	%\begin{bbox}[title={Wrapper $\mathcal{W}_{\msf{In-O(1)}} (\F)$}]
%
%Initialize $\msf{crnd} := 0$, $\msf{lastcrnd} := -1$, $\msf{runqueue} := []$
%
%\vspace{2mm} \hrule \vspace{2mm}
%
%\OnInput \inmsg{In-O(1)}{codeblock e} from $\F$:
%
%	\quad Add $e$ to $\msf{runqueue}$
%
%	\quad \Leak $e \rightarrow \mathcal{A}$
%
%\OnInput \inmsg{deliver}{idx} from $\mathcal{A}$
%
%	\quad $e \leftarrow \msf{runqueue}[idx]$
%
%	\quad Delete $\msf{runqueue}[idx]$
%
%	\quad {\bf Execute} $e$
%
%\vspace{2mm} \hrule \vspace{2mm}
%
%On every activation:
%
%	\quad $\msf{rnd} \leftarrow \F_{\msf{clock}}.\msf{clockread}$
%
%	\quad \If $\msf{rnd} \neq \msf{crnd}$:
%
%		\quad \quad $\msf{lastcrnd} \leftarrow \msf{crnd}$
%
%		\quad \quad $\msf{crnd} \leftarrow \msf{rnd}$
%
%\end{bbox}

%\begin{bbox}[title={Wrapper $\mathcal{W}_{\msf{O(1)}}$}]
%
%Initialize $\msf{crnd} := 0$, $\msf{lastcrn} := -1$, $\msf{runqueue} := []$
%
%\vspace{2mm} \hrule \vspace{2mm}
%
%\OnInput \inmsg{In O(1)}{codeblock e} from $\F$:
%
%	\quad Add $e$ to $\msf{runqueue}[\msf{crnd}+1]$ 
%
%\OnInput \inmsg{deliver}{idx} from $\mathcal{A}$:
%
%	\quad Pop $e \leftarrow \msf{runqueue}[\msf{crnd}][idx]$
%
%	\quad Execute $e$
%
%\end{bbox}

\begin{bbox}[title={$\F_{\msf{Bracha}} (\mathcal{D}, \mathcal{P} = p_1,...,p_n)$}]

See $\F_{\msf{SFE}}^{f,Rnd}$ in Katz.
%\OnInput \inmsg{input}{T} from $\mathcal{D}$ (once):
%
%	\quad \If $\msf{crnd} > 0$: \reject
%
%	\quad \Leak $T \rightarrow \mathcal{A}$
%
%	\quad \For $p_i \in \mathcal{P}$:
%
%		\quad \quad {\em In O(1)}  \Send $T \rightarrow p_i$
%
%\vspace{2mm} \hrule \vspace{2mm}
%
%\If $\msf{crnd} > 0$ and no input from $\mathcal{D}$:
%
%	\quad \For $p_i \in \mathcal{P}$:
%
%		\quad \quad {\em In O(1)} \Send $\bot \rightarrow p_i$
%
\end{bbox}

%\begin{bbox}[title={Simulator $S_{\msf{Bracha}}$}]
%
%Simulate real-world parties $\overline{\mathcal{P}} = p_1,...,p_n$ and $\Fsync{p_i}{p_j}, \forall p_i,p_j \in \overline{\mathcal{P}}$
%
%Simulate instance $\overline{\F}$ of $\F_{\msf{clock}}$.
%
%Designate same dealer $\overline{\mathcal{D}}$ as environment.
%
%Simulate dummy adversray $\mathcal{A}_{\mathcal{D}}$
%
%\vspace{2mm} \hrule \vspace{2mm}
%
%Case \#1 ( Dishonest $\mathcal{D}$ ):
%
%\OnInput \inmsg{input}{v} from $\mathcal{Z}$ for $\mathcal{D}$:
%
%	\quad \Send (input,v) $\rightarrow \mathcal{A}_{\mathcal{D}}$ {\em (Passthrough for corrupted parties in real world)}
%
%\OnInput \inmsg{m} from $\mathcal{Z}$:
%
%	\quad \Send (m) $\rightarrow \mathcal{A}_{\mathcal{D}}$
%
%\OnInput \inmsg{activates}{$p_j$} from $\F_{\msf{Bracha}}$:
%
%	\quad \If first message in round $r$:
%
%		\quad \quad Deliver messages from $\Fsync{p_j}{p_i}$ to $p_i$ through $(\msf{fetch})$ and simulate state changes.
%
%\vspace{2mm} \hrule \vspace{2mm}
%
%When protocol terminates, obtain output value $v$. Deliver $v \rightarrow \F_{\msf{Bracha}}$ as the dealer $\mathcal{D}$.
%
%\end{bbox}
%\begin{bbox}[title={Simualator $S_{\msf{Bracha}}$}]
%
%Simulate real-world parties $\mathcal{\overline{P}} = p_1,..,p_n$ anbd $\Fsync{p_i}{p_j}, \forall p_i,p_j \in \mathcal{P}$ and corrupt $t$ of them.
%
%Simulate instance $\overline{\F}$ of $\F_{\msf{clock}}$ and instance $\overline{\mathcal{W}}$ of wrapper $\mathcal{W}_{O(1)}$.
%
%Designate the same dealer $\overline{\mathcal{D}}$ as the ideal protocol.
%
%Simulate the real world adversary $\mathcal{A}$
%
%\vspace{2mm} \hrule \vspace{2mm}
%
%\OnInput \inmsg{T} from $\F_{\msf{Bracha}}$ \emph{(input to $\F_{\msf{Bracha}}$ from $\overline{\mathcal{D}}$)}:
%
%	\quad Submit $T$ to $\overline{\mathcal{D}}$
%
%	\quad Simulate state changes in all praties until $\overline{\F}.\msf{round}$ increments
%
%\OnInput \inmsg{deliver}{idx} from $\mathcal{Z}$:
%
%	\quad \Send (deliver,idx) $\rightarrow$ $\overline{\mathcal{W}}$
%
%\OnInput \inmsg{clockupdate}{$p_i$} from $\mathcal{Z}$:
%
%	\quad \If $p_i$ is corrupted: \Send (clockupdate) $\rightarrow \overline{\F}$
%
%	
%
%\end{bbox}

%	\label{fig:functionality:broadcast}
%	\caption{Ideal functionality for a byzantine broadcast. The ideal functionality is not specific to any potential real world protocol, but captures the guarantees of the broadcast. Additionally, the functionality is identical to the SFE functionality except for assumptions made for the application at hand.}
%\end{figure}
%
%The ideal functionality of for byzantine broadcast is described in Figure ~\ref{fig:functionality:broadcast}
%The functionality is cast as a secure function evaluation where the function is as described above.
%In our case we simplify the SFE functionality instead of just parmeterizing it with the function $f_{\msf{bc}}$ gives its simplicity.
%The only simplification made to SFE is that the functionality only waits for the dealer's input (in SFE this equates to assuming all other parties' inputs are already set).
%Furthermore, the function $f_{\msf{BC}}$ is put in place in the functionality.
%
%%\begin{figure}
%%	\begin{bbox}[title={\textbf{Protocol} Async-Bracha-Broadcast$(v)$}]

-- {\bf Step 0}. Input $v$ from $p$:
	
	\qquad Send $(initial,v)$ to all processes.

-- {\bf Step 1}. Wait for:
	
	\qquad one $(initial, v)$ message or $\frac{n+t}{2}$ $(echo,v)$ messages or $(t+1)$ $(ready,v)$ messages, for some $v$.

	\qquad Send $(echo,v)$ to all processes.

-- {\bf Step 2}. Wait for:
	
	\qquad $\frac{n+t}{2}$ $(echo,v)$ messages or $(t+1)$ $(ready,v)$ messages, for some $v$

	\qquad Send $(ready,v)$ to all processes.

-- {\bf Step 3}. Wait for:

	\qquad $2t+1$ $(ready,v)$ messages

	\qquad Accept $v$.

\end{bbox}

%%	\label{fig:protocol:asyncbracha}
%%	\caption{The original asynchronous broadcast protocol proposed by Bracha~\cite{bracha-broadcast}. The protocol proceeds in three rounds and terminates if enough messages are delivered.}
%%\end{figure}
%
%Next we introduce a real world protocol that realized the ideal functionality.
%We use the asynchronous broadcast primitive introduced by Bracha~\cite{bracha-broadcast} that tolerates $\frac{n}{3}$ byzantine failures.
%Below, we introduce the same broadcast protocol cast in the synchronous model.
%We modify the protocol in the following ways:
%
%\begin{itemize}
%	\item First, the protocol has to be updates to reflect the interace offered by the ideal functionality. In the ideal world, the environment gives $|\mathcal{P}|$ activations to each party in each round. In the real world protocol, on the first activation of a party in a round $r$, the party $p_i$ fetches incoming messages from previous rounds, computes the messages to be sent in this round, $\{m_{i,j,r}\}_{j}$, and uses the next $|\mathcal{P}|$ rounds to send those messages to the other parties.
%	\item Second, the protocol is modified to only accept certain messages in certain rounds. For example, the hones dealer will only accept $(input,v)$ from the environment in the first round, all parties will only accept $(VAL,v)$ messages in the second tound, and so on.
%\end{itemize}
%
%\begin{figure}[!h]
%	%\begin{bbox}[title={$\Pi_{\msf{Bracha}} (\mathcal{D}, \mathcal{P} = p_1,...,p_n)$ in $\F_{\msf{sync}}$-hybrid}]
\begin{bbox}[title={$\Pi_{\msf{Bracha}} (\mathcal{D}, \mathcal{P} = p_1,...,p_n)$ in $\F_{\msf{BD-SEC}}$-hybrid}]

Initialize $\msf{BQ} := \frac{\msf{ceil}(n+t)}{2}$, $\msf{init} := crnd$, $\msf{out} := \emptyset$

\vspace{2mm} \hrule \vspace{2mm}

{\bf Dealer $\mathcal{D}$ Protocol}

-- \OnInput \inmsg{input}{m} from $\mathcal{Z}$:

	\dquad \For $p_i \in \mathcal{P}$:

		\dquad \quad \Send $\msf{VAL}(m) \rightarrow \Fsync{\mathcal{D}}{p_i}$

\vspace{2mm} \hrule \vspace{2mm}

{\bf Party $p_i$ Protocol}

-- \OnInput \inmsg{$\msf{VAL}(m)$} from $\F_{\msf{sync},\mathcal{D},p_i}$ (once, round $\msf{init}+1$):

	\dquad \For $p_j \in \mathcal{P}$: \Send $\msf{ECHO}(m) \rightarrow \Fsync{p_i}{p_j}$ \\

 \OnInput \inmsg{$\msf{ECHO}(m)$} from $\Fsync{p_j}{p_i}$ (round $\msf{init}+2$):

	\dquad \If received $\msf{ECHO}(m)$ from $\msf{BQ}$ parties:

		\dquad \quad \For $p_j \in \mathcal{P}$: \Send $\msf{READY}(m) \rightarrow \Fsync{p_i}{p_j}$ \\

-- \OnInput \inmsg{$\msf{READY}(m)$} from $\Fsync{p_j}{p_i}$ (round $\msf{init}+3$):

	\dquad \If received $\msf{READY}(m)$ from $2t+1$ parties:

		\dquad \quad $\msf{out} := m$

		%\quad \quad \Output $m$

	%\quad \If received $\msf{ECHO}(m)$ from $\msf{BQ}-t$ parties and $\msf{READY}(m)$ from $> t$ parties and $\msf{READY}$ not sent:

	%	\quad \quad \For $p_j \in \mathcal{P}$: \Send $\msf{READY}(m) \rightarrow \Fsync{p_i}{p_j}$ \\

-- \OnInput \inmsg{output} from $\mathcal{Z}$:

	\dquad \If $\msf{out} \neq \emptyset$: \Output $\msf{out}$ 

	\dquad \Else On $j^{th}$ activation in this round:

		\dquad \quad \Send $(\msf{fetch}) \rightarrow \Fsync{p_j}{p_i}$

		\dquad \quad $m \leftarrow \Fsync{p_j}{p_i}$

\vspace{2mm} \hrule \vspace{2mm}

\If not received $2t + 1$ \msf{READY}(\textunderscore) messages by $\msf{init} + 4$:

	\dquad \Output $\bot$

\end{bbox}


%  RND 2: 
%			OnInput VAL from Dealer --> Echo
%  RND 3:
%			OnInput ECHO(m) from Pi: If 2t+1 ECHO messages --> READY
%  RND 4:
%			OnInput READY(m) from Pi: If t+1 READY --> Output m

%\end{figure}
%\begin{figure}
%	\begin{bbox}[title={Simulator $S_{\msf{Bracha}}$}]

Simulate real-world parties $\overline{\mathcal{P}} = p_1,...,p_n$ and $\Fsync{p_i}{p_j}, \forall p_i,p_j \in \overline{\mathcal{P}}$

Simulate instance $\overline{\F}$ of $\F_{\msf{clock}}$.

Designate same dealer $\overline{\mathcal{D}}$ as environment.

Simulate dummy adversray $\mathcal{A}_{\mathcal{D}}$

\vspace{2mm} \hrule \vspace{2mm}

Case \#1 ( Dishonest $\mathcal{D}$ ):

\OnInput \inmsg{input}{v} from $\mathcal{Z}$ for $\mathcal{D}$:

	\quad \Send (input,v) $\rightarrow \mathcal{A}_{\mathcal{D}}$ {\em (Passthrough for corrupted parties in real world)}

\OnInput \inmsg{m} from $\mathcal{Z}$:

	\quad \Send (m) $\rightarrow \mathcal{A}_{\mathcal{D}}$

\OnInput \inmsg{activates}{$p_j$} from $\F_{\msf{Bracha}}$:

	\quad \If first message in round $r$:

		\quad \quad Deliver messages from $\Fsync{p_j}{p_i}$ to $p_i$ through $(\msf{fetch})$ and simulate state changes.

\vspace{2mm} \hrule \vspace{2mm}

When protocol terminates, obtain output value $v$. Deliver $v \rightarrow \F_{\msf{Bracha}}$ as the dealer $\mathcal{D}$.

\end{bbox}

%\end{figure}
%
%{\bf Theorem.} {\em Protocol $\Pi_{\msf{Bracha}}$ securely realized \Fbc in the $\{\Fbdsec,\Fclock \}$-hybrid world. Assume a stateic adversary corrupted up to $\frac{n}{3}$ parties.}
%
%Consider the simulator, $\mathcal{S}$, above.
%
%If the dealer $\mathcal{D}$ is honest: In the ideal world, $\mathcal{D}$ gives input $v$ to $\F_{\msf{Bracha}}$ which gives leaks it to $\mathcal{S}$.
%The simulator submits the input to it all of the locl $\Fsync{\mathcal{D}}{p_i}$ for $p_i \in  \mathcal{P}$.
%
%$\mathcal{S}$ expects to receive $|\mathcal{P}|$ activations from $\F_{\msf{Bracha}}$ when ideal world parties attempt to read output from the functionality.
%In each activation, the simulator sufficiently ensures each party reads messages from all other parties and simualated state changes and increment the local \Fclock.
%
%The functionality waits $Rnd = 3$ rounds to deliver the output. In the first round $|\mathcal{P}|^2$ activations ensure all \msf{ECHO} messages are sent.
%In functionality round 2, activations ensure that all \msf{READY} messages are sent. The final functionality round 3, all \msf{READY}s are delivered and the simulates real world parties all output a value $v$.
%By the proof of the Bracha protocol, all real world parties output the same value. The simulator instructs 
%
%\paragraph{Typo in Katz paper}
%In synchronous protocols, parties can send at most $n = |\mathcal{P}|-1$ messages, one message to each other participant in the protocol.
%According to the functionality $\Fbdsec$, when a party sends a message, the adversary is activated by leaking information.
%This means that in one activation, each party can only send one message.
%The definition of synchronous protocols in the $\{\Fbdsec,\Fclock \}$-hybrid says that in each round each party must send (\texttt{RoundOK}) to $\Fclock$.
%Therefore, each party needs $n$ activations, $n-1$ for sending messages and 1 for sending \texttt{RoundOK}.
%Finally, any subsequent activation if $p_i$'s bit $d_i$ is still 1, $p_i$ outputs \texttt{early} to the environment.
%The ideal world needs these many acivations as well, to invoke the simulator enough to simulate the real world.
%
%The functionality as described in the paper, show in Figure~\ref{fig:sfe} allows the round to advance with only $n-1$ activations through the \texttt{output} message.
%The edit to make is to change the \texttt{activated} and \texttt{early} logic to wait until $t_i = 0$ instead of 1. That's all.
%
%
%\begin{figure}
%	\begin{bbox}[title={$\F_{\msf{sfe}}^{f,Rnd} (\mathcal{P})$}]

For each $p_i \in \mathcal{P}$ initialize $x_i = y_i := \bot$, delay $t_i := |\mathcal{P}|$.
Global $\ell := 1$

-- \OnInput \inmsg{\texttt{input}}{$v$} from \Partyi:

	\qquad Set $x_i := v$ and \Send (input, $p_i$) $\rightarrow \mathcal{A}$

-- \OnInput \inmsg{\texttt{output}} from \Partyi:

	\qquad \If $p_i \in \mathcal{H}$ and $x_i = \bot$: ignore \Else:

		\qqquad $*$ \If $t_i > 1$:

			\qqqquad Set $t_i := t_i - 1$. \If (now) $t_j = 1$ for all $p_j \in \mathcal{H}$:

				\qqqqquad Set $\ell := \ell + 1$ and $t_j := |\mathcal{P}|$ for all $p_j \in mathcal{P}$. \Send (activated,$p_i$) $\rightarrow \mathcal{A}$

		\qqquad $*$ \Else If $t_i = 1$ and $\ell < Rnd$, \Send (early) $\rightarrow p_i$

		\qqquad $*$ \Else:

			\qqqquad -- \If $x_j \neq \bot$ and for all $p_i \in \mathcal{H}$, and $y_1,...,y_n$ are not set: $r \xleftarrow{\$} R$ 
			
				\qqqqquad set $(y_1,...,y_n) := f(x_1,...,x_n, r)$

			\qqqquad -- Output $y_i$ to $p_i$

\end{bbox}

%	\label{fig:sfe}
%	\caption{The ideal functionality for secure function eveluation from Katz~\cite{synchronousuc}. It's parameterized by the function $f$ and a polynomial $Rnd$ representing the upper bound on the number of rounds the real world protocol takes. It proceeds in rounds where eact party requires $|\mathcal{P}|$ activations. And it's WRONG.}
%\end{figure}
%
%Below is the updated version with the correct number of activations.
%
%\begin{figure}
%	\begin{bbox}[title={$\F_{\msf{sfe}}^{f,Rnd} (\mathcal{P})$}]

For each $p_i \in \mathcal{P}$ initialize $x_i = y_i := \bot$, delay $t_i := |\mathcal{P}|$.
Global $\ell := 1$

-- \OnInput \inmsg{\texttt{input}}{$v$} from \Partyi:

	\qquad Set $x_i := v$ and \Send (input, $p_i$) $\rightarrow \mathcal{A}$

-- \OnInput \inmsg{\texttt{output}} from \Partyi:

	\qquad \If $p_i \in \mathcal{H}$ and $x_i = \bot$: ignore \Else:

		\qqquad $*$ \If $t_i > 0$:

			\qqqquad Set $t_i := t_i - 1$. \If (now) $t_j = 0$ for all $p_j \in \mathcal{H}$ and $\ell < Rnd$:

				\qqqqquad Set $\ell := \ell + 1$ and $t_j := |\mathcal{P}|$ for all $p_j \in mathcal{P}$. \Send (activated,$p_i$) $\rightarrow \mathcal{A}$

		\qqquad $*$ \Else If $t_i = 0$ and $\ell < Rnd$, \Send (early) $\rightarrow p_i$

		\qqquad $*$ \Else:

			\qqqquad -- \If $x_j \neq \bot$ and for all $p_i \in \mathcal{H}$, and $y_1,...,y_n$ are not set: $r \xleftarrow{\$} R$ 
			
				\qqqqquad set $(y_1,...,y_n) := f(x_1,...,x_n, r)$

			\qqqquad -- Output $y_i$ to $p_i$

\end{bbox}

%	\label{fig:sfe}
%	\caption{Same ideal functionality as Figure~\ref{fig:sfe} but corrected}
%\end{figure}
%
%\begin{figure}
%	\begin{bbox}[title={\textbf{Wrapper} $\mathcal{W}_{\msf{katz}} ( \pi )$}]

Initialize $\msf{todo} := \emptyset, t := 0, \msf{output} := \bot, \msf{ready} := 0, \msf{round} := 1$

\vspace{2mm} \hrule \vspace{2mm}

-- \OnInput \inmsg{input}{$x$} from $p_i$:  Forward \inmsg{input}{$x$} to $\pi$

-- \OnInput \inmsg{output} from $\mathcal{Z}$

	\qquad \If $t > 0$: Set $t = t-1$  If $\msf{todo} \neq \emptyset$:

		\qqquad Pop $(pid, msg) \leftarrow \msf{todo}$ and execute \Send $(msg) \rightarrow \Fbdsec$ where $j = pid$ and $i$ is the pid of $\pi$.

	\qquad \Else \If $t == 0$: Set $t = t-1$ and \Send $(\texttt{RoundOK}) \rightarrow \Fclock$

	\qquad \Else:
	
		\qqquad \If $\msf{ready}$: \Output \msf{output}

		\qqquad \Else \Output $(\texttt{early})$

\vspace{2mm} \hrule \vspace{2mm}

-- \OnInput \inmsg{Send}{pid}{msg} from $\pi$:

	\qquad Append $(pid, msg)$ to $\msf{todo}$

\end{bbox}

%	\label{fig:wrapper}
%\end{figure}
%
%\newpage

\subsection{Broadcast with Synchronous Wrapper}
This section descibres how the synchronous wrapper works.
The example is bracha broadcast with the functionalities, protocols, simulator and wrapper listed below.

\paragraph{Synchronous Wrapper}
The synchronous wrapper provide synchrhonous communication functionality to all ITMs in both the real and ideal worlds.
At a high level the wrapper allows arbitrary code blocks to be executed in a synchronous way.
It allows for functionality and protocol code to be simpler and easier to understand with intuitive abstractions. 
We demonstrates this aspect of the wrapper through the a synchronous broadcast protocol that we provide a full UC construction for.
First, we describe the wrapper in Figure~\ref{fig:wrapper:synchronous}.

%The synchronous wrapper handles delaying execution of codeblocks within a upper bound $\Delta$ that is fixed by the calling protocol. 
%The synchronous wrapper provides a simple interface to the parties and functionalities, defined in Figure~\ref{fig:wrapper:synchronous}.
%The simplest example to demonstrate how the wrapper works is with a simple synchronous, point-to-point channel, $\F_{sync-chan}$ (Figure~\ref{fig:functionality:channel}).

An ITM wishing to execute a codeblock within $O(\Delta)$ round, sends the codeblock to the wrapper and the maximum delay $\Delta$ to be imposed on its execution.
The wrapper assigns the maximum, $\Delta$, delay and adds it to its internal queue.
It then leaks the round and the index in the run queue to the adversary and increments an internal $delay$ counter.
The $delay$ variable is incremented whenever a new codeblock is scheduled to run and is decremented when the environment sends \texttt{poll} to the wrapper.

Schedule blocks can be forced to execute by the environment \emph{and} and the adversary.
It's necessary for the environment to be able to force the wrapper to make progress in order to ensure \emph{guaranteed termination}.
The adversary can choose when to execute a code block by sending it and \texttt{exec} message and the location in the runqueue to execute.
The environment can continuously call \texttt{poll} to force the next codeblock to execute.
If there are no remaining code blocks in round $r$ on \texttt{poll}, the wrapper jumps ahead to the next round $r' > r$ in which there exists a scheduled code block.
When \texttt{poll} is sent to the wrapper, the adversary is also activated so that it can simulate \texttt{poll} in its simulated real-world wrapper.

In Figure~\ref{fig:functionality:channel}, we should a simple channel functionality that uses the wrapper to schedule messages in a synchronous fashion.
In a protocol execution with pair-wise channels, $\F_{\msf{sync-chan},s,r}$, the adversary only needs to communicate with the wrapper~\footnote{In other adversarial models and un-authenticated channels, the adversary would still need to use the channels to modify messages.}.

\begin{figure}[!htb]
\begin{subfigure}{\textwidth}
	\begin{bbox}[title={$\F_{\msf{sync-chan}(p_s, p_r, r, \Delta)}$}]

\OnInput \inmsg{send}{msg} $d \token$ from $p_s$:
	\begin{renumerate}
	\item In $O(\Delta): ~\Send (msg, d \token)$
	\end{renumerate}
\end{bbox}

	\caption{A synchronous channel written in the style of the synchronous wrapper. The ``In $O(\Delta)$'' is high level code that is expanded in the next figure.}
	\label{fig:functionality:channel}
\end{subfigure}
\begin{subfigure}{\textwidth}
	\begin{bbox}[title={$\F_{\msf{sync-chain}}(p_s, p_r, r, \Delta)$}]

\OnInput \inmsg{send}{msg $d \token$} from $p_s$:
	\begin{renumerate}
	\item \Send (\msf{Leak}, (\msf{Send}, $msg$, $p_s$, $p_r$)) $\rightarrow \mathcal{W}_{sync}$
 	\item Expect $\msf{OK} \leftarrow \mathcal{W}_{sync}$
	\item \Send (\msf{schedule}, $\{\Send msg \rightarrow p_i\}$) $\rightarrow \mathcal{W}_{async}$
	\item Expect $\msf{OK} \leftarrow \mathcal{W}_{sync}$
	\end{renumerate}
\end{bbox}

	\caption{The high level invocation of ``In $O(\Delta)$'' is expanded into a ``schedule'' msg that is sent to the wrapper.}
\end{subfigure}
\end{figure}

\begin{figure}[!htb]
	\begin{bbox}[title={\textbf{Wrapper} $\mathcal{W}_{\msf{sync}}$}]

Initialize $\msf{leakbuffer} := \emptyset, \msf{runqueue} := \emptyset, r := 0, delay := 0$

\vspace{2mm} \hrule \vspace{2mm}

%\begin{itemize}[before=\setlength{\baselineskip}{20pt}, itemsep=-2ex]

\OnInput \inmsg{schedule}{codeblock $e$, $\delta$} $1 \token$ from $\mathcal{F}_i$/$P_i$:

	\begin{renumerate}

	\item $idx \leftarrow$ append $e$ to $\msf{runqueue}[r+ \delta]$

	\item $delay = delay + 1$

	\item Append (schedule, $r + \delta$, $idx$) to \msf{leakbuffer}

	\item \Send (scheduled) $\rightarrow \mathcal{F}_i/P_i$
	\end{renumerate}

\OnInput \inmsg{delay} $d \token$ from $\mathcal{A}$:

	\begin{renumerate}
	\item $delay = delay + d$
	\end{renumerate}

\OnInput \inmsg{execute}{$rnd$}{$idx$} from $\mathcal{A}$:
	\begin{renumerate}	
	\item $e \leftarrow$ pop $\msf{runqueue}[r][idx]$
	\item Execute $e$
	\end{renumerate}

\OnInput \inmsg{leak}{msg} from $\mathcal{F}_i/P_i$:
	\begin{renumerate}
	\item Append $(msg, \mathcal{F}_i/P_i, d \token)$ to $\msf{leakbuffer}$
	\end{renumerate}

\OnInput \inmsg{poll} $1 \token$ from $\mathcal{Z}$:
	\begin{ritemize}
	\item If $delay > 0$:
		\begin{renumerate}
		\item $delay = delay - 1$
		\item\Send (poll) to $\mathcal{A}$
		\end{renumerate}
	\item Else: 
		\begin{renumerate}
		\item $rnd \leftarrow \msf{argmin}_{r}\{ \msf{runqueue}[r] \neq \emptyset \}$
		\item $e \leftarrow$ pop $\msf{runqueue}[rnd]$
		\item Execute $e$
		\end{renumerate}
	\end{ritemize}
\OnInput \inmsg{round} from $\mathcal{F}_i/P_i$:
	\begin{renumerate}
	\item \Send $r \rightarrow \mathcal{F}_i/P_i$
	\end{renumerate}
\OnInput \inmsg{getleaks} from $\mathcal{A}$:
	\begin{renumerate}
	\item $d \leftarrow$ total import in $\msf{leakbuffer}$
	\item \Send $(\msf{leakbuffer}, d \token) \rightarrow \mathcal{A}$
	\item $\msf{leakbuffer} = \emptyset$
	\end{renumerate}
%\end{itemize}

\end{bbox}

	\caption{This wrapper implements the enhancements made to the protocol to conform to the interace of the ideal functionality \Fbc. It reads in incoming messages in the first activation of a round and uses the next $|\mathcal{P}|$ activations to send messages to others.}
	\label{fig:wrapper:synchronous}
\end{figure}

\paragraph{A Simple Broadcast Protocol}
We present another example of the synchronous wrapper with a simple synchronous broadcast protocol.
The ideal functionality for a synchronous broadcast protocol is shown in Figure \ref{fig:functionality:broadcast_import}
The functionality just accepts input from the dealer (fixed to be pid = 1) and tells the wrapper to deliver the outputs of each of the parties \emph{within} $O(\Delta)$.
This functionality, though, abstracts away the details of sending a message to the wrapper with the statement ``in $O(4\Delta)$''.
Figure~\ref{fig:functionality:broadcast_import_real} shows the exact message being exchanged with the wrapper.

\begin{figure}[!htb]
\begin{subfigure}{\textwidth}
	
\begin{bbox}[title={$\F_{\msf{bcast}} (\mathcal{D}, \mathcal{P}=p_1,...,p_n)$}]

\OnInput \inmsg{input}{v}, $n(4n+1) \token$ from $\mathcal{D}$:
	\begin{renumerate}
	\item For $p_i \in \mathcal{P}$:
		\begin{ritemize}
		\item In $O(4 \Delta)$: $\{\Send v \rightarrow p_i\}$
		\end{ritemize}
	\item \Leak (input, $v$), $n(4n+1) \token$ $\rightarrow \mathcal{A}$
	\end{renumerate}

\end{bbox}

	\caption{A simple broadcast functionality that delivers the output to all the other parties within $O(\Delta)$ rounds. The functionality is described at a high level, with more specific messages and wrapper interaction shown below.}
	\label{fig:functionality:broadcast_import}
\end{subfigure}
\newline
\begin{subfigure}{\textwidth}
	\begin{bbox}[title={$\F_{\msf{bcast}} (\mathcal{D}, \mathcal{P}=p_1,...,p_n)$}]

-- \OnInput \inmsg{input}{v}, $n(4n+1) \token$ from $\mathcal{D}$:

	\qquad For $p_i \in \mathcal{P}$:
		
		\qqquad \Send (schedule, $\{\Send v \rightarrow p_i\}$, $4 \Delta$) $\rightarrow \mathcal{W}_{sync}$

	\qquad \Send (leak, (input, $v$), $n(4n+1) \token$) $\rightarrow \mathcal{W}_{sync}$

\end{bbox}

	\caption{This broacsat functionality illustrates the actual messages being passed between the functionality above and the wrapper. The wrapper handles synchronous delivery of messages {\em and} leaks.}
	\label{fig:functionality:broadcast_import_real}
\end{subfigure}
\end{figure}

%\begin{figure}[!htb]
%\begin{subfigure}{\textwidth}
%	\begin{bbox}[title={$\F_{\msf{async-bcast}} (\mathcal{D}, \mathcal{P}=p_1,...,p_n)$}]

\OnInput \inmsg{input}{v}, $n(4n+1) \token$ from $\mathcal{D}$:
	\begin{renumerate}
	\item For $p_i \in \mathcal{P}$:
		\begin{ritemize}	
		\item \Eventually $\{\Send v \rightarrow p_i\}$ 
		\end{ritemize}

	\item \Leak (input, $v$), $n(4n+1) \token$ $\rightarrow \mathcal{A}$
	\end{renumerate}

\end{bbox}


%	\caption{A simple broadcast functionality that delivers the output to all the other parties {\em eventually}. The functionality is described at a high level, with more specific messages and wrapper interaction shown below.}
%	\label{fig:functionality:broadcast_import}
%\end{subfigure}
%\newline
%\begin{subfigure}{\textwidth}
%	\begin{bbox}[title={$\F_{\msf{async-bcast}} (\mathcal{D}, \mathcal{P}=p_1,...,p_n)$}]

\OnInput \inmsg{input}{v}, $n(4n+1) \token$ from $\mathcal{D}$:
	\begin{renumerate}
	\item For $p_i \in \mathcal{P}$:
		\begin{ritemize}	
		\item \Send (schedule, $\{\Send v \rightarrow p_i\}$) $\rightarrow \mathcal{W}_{async}$
		\end{ritemize}

	\item \Leak, (input, $v$), $n(4n+1) \token$ $\rightarrow \mathcal{W}_{async}$
	\end{renumerate}

\end{bbox}


%	\caption{This broacsat functionality illustrates the actual messages being passed between the functionality above and the wrapper. The wrapper handles asynchronous delivery of messages {\em and} leaks.}
%	\label{fig:functionality:broadcast_import_real}
%\end{subfigure}
%\end{figure}
A protocol realizing this functionality was proposed by Bracha~\cite{bracha} in the asynchronous setting.
%The pseudocode of the original protocol without any UC-specific details is shown in Figure \ref{fig:protocol:asyncbracha}.
%
%\begin{figure}[!htb]
%\begin{minipage}[t]{0.55\textwidth}
%	\begin{bbox}[title={\textbf{Protocol} Async-Bracha-Broadcast$(v)$}]

-- {\bf Step 0}. Input $v$ from $p$:
	
	\qquad Send $(initial,v)$ to all processes.

-- {\bf Step 1}. Wait for:
	
	\qquad one $(initial, v)$ message or $\frac{n+t}{2}$ $(echo,v)$ messages or $(t+1)$ $(ready,v)$ messages, for some $v$.

	\qquad Send $(echo,v)$ to all processes.

-- {\bf Step 2}. Wait for:
	
	\qquad $\frac{n+t}{2}$ $(echo,v)$ messages or $(t+1)$ $(ready,v)$ messages, for some $v$

	\qquad Send $(ready,v)$ to all processes.

-- {\bf Step 3}. Wait for:

	\qquad $2t+1$ $(ready,v)$ messages

	\qquad Accept $v$.

\end{bbox}

%	\caption{The original asynchronous broadcast protocol proposed by Bracha~\cite{bracha-broadcast}. The protocol proceeds in three rounds and terminates if enough messages are delivered.}
%	\label{fig:protocol:asyncbracha}
%\end{minipage}
%\end{figure}
%
%The protocol is simple and designed to handle $t < \frac{n}{3}$ Byzantine failures.
%The parties stage an input for commit if they witness a Byzantine qourum ($\frac{n+t}{2}$) of \texttt{ECHO} messages and send out \texttt{READY} messages to all of the other parties.
%Finally, each party waits for $2t + 1$ \texttt{READY} messages (at least 1 \texttt{READY} message from an honest party) and commits to the value.

We use this protocol in our synchronous model in Figure~\ref{fig:prot:bracha_ours}.
The real world execution of this protocol includes some set of honest parties running the protocol and pairwise authenticated synchronous channels $\F_{\msf{sync-chan},s,r}$ that are unique to each round (i.e. each such functionality is one-shot).
The channels are described above in Figure~\ref{fig:functionality:channel}.

\begin{figure}[!htb]
	\begin{bbox}[title={$\Pi_{\msf{Bracha}} (\mathcal{D}, \mathcal{P} = p_1,...,p_n)$ in $\F_{\msf{sync-chan}}$-hybrid}]

Initialize $\msf{BQ} := \frac{\msf{ceil}(n+t)}{2}$, $\msf{init} := crnd$, $\msf{out} := \emptyset$

\vspace{2mm} \hrule \vspace{2mm}

% dealer input INPUT
{\bf Dealer $\mathcal{D}$ Protocol}

\OnInput \inmsg{input}{m}{$ n(4n+1) \token $} from $\mathcal{Z}$:
	\begin{renumerate}
	\item \For $p_i \in \mathcal{P}$:

		\quad  \Send $\msf{VAL}(m),4n \token \rightarrow \Fsync{\mathcal{D}}{p_i}$
	\end{renumerate}

\vspace{2mm} \hrule \vspace{2mm}

{\bf Party $p_i$ Protocol}

% on input VAL
\OnInput \inmsg{$\msf{VAL}(m)$}{$4n \token$} from $\Fchan{\mathcal{D}}{p_i}$ (once):
	\begin{renumerate}
	\item \For $p_j \in \mathcal{P}$: 
	
	\quad \Send $\msf{ECHO}(m), 3 \token \rightarrow \Fchan{p_i}{p_j}$
	\end{renumerate}

\OnInput \inmsg{$\msf{ECHO}(m)$}{$3 \token$} from $\Fchan{p_j}{p_i}$:
	\begin{renumerate}
	\item \If received $\msf{ECHO}(m)$ from $\msf{BQ}$ parties:
		\begin{ritemize}
		\item \For $p_j \in \mathcal{P}$: 
		
		\quad \Send $\msf{READY}(m), 0 \token \rightarrow \Fchan{p_i}{p_j}$ \\
		\end{ritemize}
	\end{renumerate}
% on input READY

\OnInput \inmsg{$\msf{READY}(m)$}{$0 \token$} from $\Fchan{p_j}{p_i}$:
	\begin{renumerate}
	\item \If received $\msf{READY}(m)$ from $2t+1$ parties:
		\begin{ritemize}
		\item $\Send m \rightarrow \mathcal{Z}$
		\end{ritemize}
	\end{renumerate}

\vspace{2mm} \hrule \vspace{2mm}

\end{bbox}


	\caption{The Bracha broadcast protocol written in the UC framework with access to pairwise authenticated channel functionalities $\Fchan{s,r}$.}
	\label{fig:prot:bracha_ours}
\end{figure}

\paragraph{Bracha Simulator}
Realizing the broadcast functionality with the Bracha's broadcast protocol requires a simulator that can satisfies
\[ \text{EXEC}_{\mathcal{F_{\msf{bcast}}},\mathcal{S},\mathcal{Z}} \approx \text{EXEC}_{\Pi_{\msf{Bracha}},\mathcal{D},\mathcal{Z}} \]
for all $\mathcal{Z}$.

First, we describe the simulator at a high level.
The simulator internally runs a copy of the real world environment with its own synchronous wrapper $\mathcal{W}_{sync}'$.
For every ideal party $\phi_i$ in the ideal world the simulator creates a party $\pi_i$ in its simulation.
Finally it simulates the pairwise channels that are used throughout the execution.

As expected, the simulator forwards all messages from $\mathcal{Z}$ to any corrupted parties $p_i$ and vice versa.
On every activation, the simulator requests leaks from the ideal world wrapper.
It checks for newly schedules code blocks and maintain and internal copy of the run queue and $delay$ variable from the ideal world wrapper $\mathcal{W}_{sync}$
If the dealer is honest, the eventually fetch the dealer's leaked input from the wrapper. 
It replays the honest dealer's input and emulates any subsequent activations and state changes.

Finally, the simulator asks $\mathcal{W}_{sync}'$ for new leaks and does the following:
\begin{enumerate}
	\item It stores the leaks and returns them to the environment when asked for it.
	\item If $n$ new codeblocks have been scheduled in the simulation, it adds $n$ delay to the ideal world wrapper. The purpose of adding $delay$ to the ideal world is to ensure that the environment can't force output being delivered to a party in the ideal world before output is delivered in the simulation.
\end{enumerate}

When $\mathcal{Z}$ sends \texttt{poll} to $\mathcal{W}_{sync}$ the simulator is also activated with a \texttt{poll} message.
The simulator forwards this message to its internal wrapper and emulates any further activations of state changes.
It requests leaks from $\mathcal{W}_{sync}'$ and repeats the steps outlines above.
Finally, if any simulated party $\pi_i$ outputs a value, the simulator executes the corresponding codeblock for the ideal party $\phi_i$.

One case to handle is if the dealer is corrupt and never gives input to the ideal functionality.
Instead, the environment sends parties on behalf of the corrupt dealers through the simulator.
In this case, the simulator will never receive any leaked input from $\F_{bcast}$ but may still receive output from simulated honest parties.
If an output is received, the simulator acts on behalf of the corrupt dealer, sends the input to $\F_{bcast}$, gets leaks from $\mathcal{W}_{sync}$, and executes the corresponding codeblock.

\begin{figure}[!htb]
	\begin{bbox}[title={Simulator $\mathcal{S}_{bracha} (\mathcal{D}, \mathcal{P}, \Delta)$}]

Simulate real world parties $p_1',...,p_n'$ and the simulated dealer $\mathcal{D}'$.

Initialize $\msf{idealqueue} := \emptyset, idealdelay := 0$

\vspace{2mm} \hrule \vspace{2mm}

\OnInput \inmsg{\texttt{get-leaks}} from $\mathcal{Z}$:
	\begin{renumerate}
	\item $leaks \leftarrow$ \{ \Send (\texttt{get-leaks}) $\rightarrow \mathcal{W}_{sync}$\}

	\item For $l \in leaks$:
		\begin{renumerate}
		\item If $l$ is (input, v), $n(4n+1) \token$ from $\F_{bcast}$:

			\quad Simulate (input, $v$, $n(4n+1) \token$) $\rightarrow \mathcal{D}'$ 

		\item Else If $l$ is (schedule, $rnd$, $idx$) from $\F_{bcast}$:

			\quad Map $p_i$ to $(rnd,idx)$ for the $i$th such leak.

			\quad $idealdelay = idealdelay + 1$

		\item Else:

			\quad $idealdelay = idealdelay + 1$
		\end{renumerate}
	%\qquad $leaks \leftarrow$ \{ \Send (\texttt{get-leaks}) $\rightarrow \mathcal{W}_{sync}'$\}
	\item $leaks \leftarrow \msf{SimGetLeaks}$

	\item \Send $leaks \rightarrow \mathcal{Z}$
	\end{renumerate}

\OnInput \inmsg{poll} from $\mathcal{Z}$:
	\begin{renumerate}
	\item execute \msf{Poll}
	\end{renumerate}

\OnInput \inmsg{delay}{$d \token$} from $\mathcal{Z}$:
	\begin{renumerate}
	\item Simulate $(delay, d \token) \rightarrow \mathcal{W}_{sync}'$

	\item \Send $(\texttt{delay}, d \token) \rightarrow \mathcal{W}_{sync}$

	\item $idealdelay = idealdelay + d$
	\end{renumerate}

\OnInput \inmsg{exec}{$rnd$}{$idx$} from $\mathcal{W}_{sync}$:
	\begin{renumerate}
	\item Simulate $(\texttt{exec}, rnd, idx) \rightarrow \mathcal{W}_{sync}'$

	\item Forward any messages from a simulated part $p_i'$ or $\mathcal{A}'$
	\end{renumerate}

\end{bbox}

	\caption{The simulator for the bracha protocol in the synchronos world.}
	\label{fig:sim:bracha_ours}
\end{figure}

\begin{figure}[!htb]
	\begin{subfigure}{\textwidth}
	\begin{bbox}[title={Algorithm $\msf{SimGetLeaks}$}]

$leaks \leftarrow$ \{ \Send (\texttt{get-leaks}) $\rightarrow \mathcal{\mathcal{A}'}$ \}

$n \leftarrow \#$ of (schedule,...) messages in $leaks$

$idealdelay = idealdelay + n$

\{ \Send (delay, $n \token$) $\rightarrow \mathcal{W}_{sync}$ \}

\end{bbox}



	\label{fig:algo:simgetleaks}
	\caption{Function to get leaks from the simulated real world and update the delay of the ideal world if new codebloks have been schedules in the simulated wrapper, $\mathcal{W}_{sync}'$.}
	\end{subfigure}
	\newline
	\begin{subfigure}{\textwidth}
	\newcommand{\idealdelay}{{\color{Blue} \msf{idealdelay} }}

\begin{bbox}[title={Algorithm $\msf{Poll}$}]

\begin{renumerate}

  	\item $\idealdelay \minuseq 1$
  	
  	\item If $\idealdelay = 0$:
  	 
  		\quad \Send $(\texttt{delay}, 1 \token) \rightarrow \mathcal{W}_{sync}$

  		\quad $\idealdelay = 1$

  	\item \Send $(\texttt{poll},) \rightarrow \mathcal{W}_{sync}'$
 
  	\item If output $m$ from party $P_i'$:

			\quad Call $\msf{SimGetLeaks}$

			\quad Call \msf{SimPartyOutput}(m, $P_i'$)
		
		Else if $m$ from $\mathcal{A}'$:

			\quad Send $m \rightarrow \mathcal{Z}$

\end{renumerate}

\end{bbox}

	\label{fig:algo:poll}
	\caption{Function that forwards a (\texttt{poll}) message to the simulated wrapper and waits for a message back from a simulated party or the adversary. If an output is given from a party and the fnuctionality, $\F_{bcast}$ functionality did not leak an input, the dealer is corrupt. Therefore, the simulator gives input to $\F_{bcast}$ and delivers that party's output.}
	\end{subfigure}
\end{figure}
