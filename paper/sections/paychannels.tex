\subsection{Unidirectional Payment Channels}

In the unidirectional channel there are two parties, $P_s$ and $P_r$, where only the ``sender'' $P_s$ is sending payments to $P_r$.
The incentive structure in this case is simpler than bi-directional payments because one of the two paties, the receiver, has an incentive to always close with the most recent state of the channel.
Therefore, we consider the receiver's request to close the channel a ``cooperative close''. 
When the sender wants to close the channel, it can tell the receiver and submit a state signed by both of them---also a ``cooperative close''.
Otherwise, it submits a balance signed by only itself and initiates an ``uncooperative close''.

Assumptions:
\begin{itemize}
  \item The channels are already open and the protocol is parameterized by the initial balances of the two parties
  \item ``on chain'' operations like close and settle take O(delta) rounds through a smart contract and atomic broadcast
  \item ``off chain'' communication is instant, i.e. O(1) rounds to send message
  \item contract outputs are broadcast immediately
  \item ``cooperative close'' is only one on-chain operation, so takes O(delta)
  \item ``uncooperative close'' is (1. sender submits state and starts disute, 2. receiver can counter with later state and channel is settled): O(2*delta)
\end{itemize}


%\begin{figure}[!htb]
%	\begin{bbox}[title={$\F_{\msf{abc}} (C, P_s, P_r, \Delta)$}]

Initialize $buf = \{\}$

\vspace{2mm} \hrule \vspace{2mm}

\OnInput \inmsg{bcast}{msg} from $P_i$:
	\begin{renumerate}
		\item Append $(msg, P_i)$ to $buf$
		\item For $p_i$ in $P_s,P_r$:
		\begin{renumerate}
			\item \msf{codeblock} = \{
			
				\quad Send $buf \rightarrow p_i$

			\}
			\item Send (schedule, \msf{codeblock}, $\Delta$) $\rightarrow \mathcal{W}_{sync}$
		\end{renumerate}
	\end{renumerate}
\end{bbox}

\begin{bbox}[title={$\mathcal{C}_{pay}(P_s,P_r, \msf{balances}, \Delta)$}]

Intialize $T_{settlement} := 2 \Delta$, $T_{deadline} := 0$, $\msf{nonce} := 0$,


$\msf{state} := (\msf{balances}[P_s], \msf{balances}[P_r], \msf{nonce})$, $\msf{FLAG} := \msf{OFFCHAIN}$

\vspace{2mm} \hrule \vspace{2mm}

\OnInput \inmsg{close}{\msf{state'}}{$\msf{sig}_s$}{$\msf{sig}_r$} from $P_i$:
	\begin{renumerate}
		\item Assert $\msf{flag} = \msf{OFFCHAIN}$
		\item Assert $\msf{CheckSig}(P_s,\msf{sig}_s, \msf{state}')$
		
		\item $(b_s, b_r, n) \leftarrow \msf{state}'$

		\item Set $\msf{nonce} = n$, $\msf{state} = \msf{state}'$

		If $\msf{CheckSig}(P_r, \msf{sig}_r, \msf{state}')$:
		\begin{renumerate}
			\item Set $\msf{flag} = \msf{Closed}$
			\item Send (\msf{Closed}, \msf{state}') $\rightarrow \F_{\msf{abc}}$ 
		\end{renumerate}

		Else:
		\begin{renumerate}
			\item Set $\msf{flag} = \msf{UnCoopClose}$ 
			\item Set $T_{deadline} = T_{now} + T_{settleent}$
			\item Send (\msf{UnCoopClose}, \msf{state}', $T_{deadline}$) $\rightarrow \F_{\msf{abc}}$
		\end{renumerate}
		
	\end{renumerate}

\OnInput \inmsg{challenge}{\msf{state}'}{$\msf{sig}_s$}{$\msf{sig}_r$} from $P_i$:
	\begin{renumerate}
		\item Assert $\msf{flag} = \msf{UnCoopClose}$

		Assert $\msf{CheckSig}(P_s, \msf{sig}_s, \msf{state}')$

		Assert $\msf{CheckSig}(P_r, \msf{sig}_r, \msf{state}')$
		\item $(b_s, b_r, n) \leftarrow \msf{state}'$
		\item Assert $n > \msf{nonce}$

		\item Set $\msf{flag} = \msf{Closed}$
		
		Set $\msf{state} = \msf{state}'$
		\item Send (\msf{Closed}, \msf{state}') $\rightarrow \F_{\msf{abc}}$
	\end{renumerate}
\end{bbox}

\begin{bbox}[title={$\F_{\msf{off-chain-chan}}(P_s, P_r)$}]

\OnInput \inmsg{send}{msg} from $P_s$:
	\begin{renumerate}
			\item \msf{codeblock} = \{
			
				\quad Send msg $\rightarrow P_r$

			\}
			\item Send (schedule, \msf{codeblock}, 1) $\rightarrow \mathcal{W}_{sync}$
	\end{renumerate}

\end{bbox}

%	\caption{Permissioned atomic broadcast parameterized by contract code $C$.}
%\end{figure}
\begin{figure}[!htb]
	\begin{bbox}[title={$\mathcal{W}_{contract} (\mathcal{P}=P_1,...,P_n, \mathcal{C}, \Delta)$}]

\underline{Incoming messages from parties:}
\vspace{2mm}

\OnInput \inmsg{msg} from $P_i$:
	\begin{renumerate}
		\item In $O(\Delta)$ rounds:
			
			\quad \Send (msg) $\rightarrow \mathcal{C}$
		\item \Leak (msg)
	\end{renumerate}

\underline{Outgoing from $\mathcal{C}$}
\vspace{2mm}

\OnInput \inmsg{{\bf Broadcast} msg} from $\mathcal{C}$:
	\begin{renumerate}
		\item \Leak (msg)
		\item For $p_i$ in $\mathcal{P}$:
		\begin{renumerate}
			\item In $O(1)$ rounds:

			\quad \Send (msg) $\rightarrow p_i$
		\end{renumerate}
	\end{renumerate}
\end{bbox}

\end{figure}

\begin{figure}[!htb]
	\begin{bbox}[title=$\Pi_{pay}$: $\msf{Contract_{pay}}$]

\Init $(\msf{P_L}, \msf{P_R})$:

\quad $\msf{deposits}_L, \msf{deposits}_R := 0$

% deposit($X)
\OnInput \inmsg{deposit}(tx) from \Partyi:

	\quad $\msf{deposits}_i += \msf{tx.value}$

	\quad $\msf{out}(\msf{deposits}_L, \msf{deposits}_R)$

% aux_out 
\OnInput \inmsg{output}{\msf{aux_{out}}}{\msf{tx}}:

	\quad parse $\msf{aux_{out}}$ as $(\msf{wd_L},\msf{wd_R})$

	\quad \For $i \in \{L,R\}$: $\msf{send}(P_i, \msf{wd_i})$

\end{bbox}

\end{figure}

\begin{figure}[!htb]
	\begin{bbox}[title={$\F_{\msf{pay}} (P_s, P_r, \msf{balances}, \Delta)$}]

Initialize $\msf{flag} := \msf{OPEN}$

\vspace{2mm} \hrule \vspace{2mm}

\OnInput \inmsg{pay}{v} from $P_s$:
	\begin{renumerate}
			\item In $O(1)$ rounds:  
				
				\quad \Require $\msf{flag} = \msf{OPEN}$
				
				\quad If $\msf{balances}[P_s] >= v$:
			
				\qquad $\msf{balances}[P_s] -= v$
				
				\qquad $\msf{balances}[P_r] += v$
				
				\qquad Send $(\msf{pay}, v) \rightarrow P_r$

		\item Leak $(\msf{pay}, v)$

	\end{renumerate}

\OnInput \inmsg{close} from $P_i$:
	\begin{renumerate}	
		\item If $P_i = P_r$ or ($P_i = P_s$ and $P_s$ is honest):
			\begin{renumerate}
			\item In $O(\Delta)$ rounds:  

			\qquad \Require $\msf{flag} = \msf{OPEN}$

			\qquad In $O(1)$ rounds:  Send $(\msf{close}, b_r, b_s) \rightarrow P_s$

			\qquad In $O(1)$ rounds: Send $(\msf{close}, b_r, b_s) \rightarrow P_r$

			\end{renumerate}

		Else:
		\begin{renumerate}
			\item In $O(k \times \Delta)$ rounds:

			\qquad \Require $\msf{flag} = \msf{OPEN}$

			\qquad In $O(1)$ rounds:  Send $(\msf{close}, b_r, b_s) \rightarrow P_s$

			\qquad In $O(1)$ rounds:  Send $(\msf{close}, b_r, b_s) \rightarrow P_r$

		\end{renumerate}
		
		\item Leak $(\msf{close}, b_r, b_s, P_i)$
		
	\end{renumerate}

\end{bbox}

\end{figure}

\begin{figure}[!htb]
	\begin{bbox}[title={$\Pi_{\msf{pay}} (P_s, P_r, b_s, b_r)$}]

Initialize $\msf{state} := (b_s, b_r, 0)$, $\msf{nonce} := 0$, $\msf{flag} := \msf{OPEN}$

$\sigma_{sender} = \bot$

\vspace{2mm} \hrule \vspace{2mm}

\underline{Sender $P_s$}:

\OnInput \inmsg{pay}{$v$} from $\mathcal{Z}$:
	\begin{renumerate}
	\item \Require $b_s \geq v$
	\item $b_s \minuseq v$; $b_r \pluseq v$; $\msf{nonce} \pluseq 1$
	\item $\msf{state} = (b_s, b_r, \msf{nonce})$
	\item $OK \leftarrow $ \{ \Send $(\msf{pay}, \msf{state}, \sigma) \rightarrow \F_{\msf{chan}}$\}
	\end{renumerate} 
	\quad \Send $OK \rightarrow \mathcal{Z}$	

\OnInput \inmsg{\msf{Close}}{\msf{state}'} from $\F_{contract}$
	\begin{renumerate}
		\item Set $\msf{flag} = \msf{CLOSE}$
		\item Set $\msf{state} = \msf{state}'$
	\end{renumerate}

\

\underline{Receiver $P_r$}:

\OnInput \inmsg{pay}{\msf{state}'}{$\sigma_{sender}'$} from $\F_{chan}$
	\begin{renumerate}
		\item \Require $\msf{flag} = \msf{OPEN}$
		\item $(b_s', b_r', \msf{nonce}') \leftarrow \msf{state}'$
		\item \Require $b_s' \geq 0$ \\
			  \Require $\msf{nonce'} = \msf{nonce} + 1$

		\item $v \leftarrow b_s - b_s'$
		\item \Require $\msf{CheckSig}(P_s, \sigma_{sender}, \msf{state}')$ \\
			  \Require $b_s \geq v$

		\item Set $\msf{state}' = \msf{state}$, $\sigma_{sender} = \sigma_{sender}'$, $\msf{nonce} = \msf{nonce}'$
	\end{renumerate}

\OnInput \inmsg{\msf{UnCoopClose}}{\msf{state}'}{$T_{deadline}$} from $\F_{chan}$:
	\begin{renumerate}
		\item \Require $\msf{flag} = \msf{OPEN}$
		\item $(b_s', b_r', \msf{nonce}') \leftarrow \msf{state}'$
		\item If $\msf{nonce}' < \msf{nonce}$:

			\quad Send $(\msf{challenge}, \msf{state}, \sigma_{sender}) \rightarrow \F_{contract}$

		\item Set $\msf{flag} = \msf{CLOSE}$
	\end{renumerate}

\

\underline{All $P_i$:}

\OnInput \inmsg{close} from $\mathcal{Z}$:
	\begin{renumerate}
		\item \Require $\msf{flag} = \msf{OPEN}$
		\item Set $\msf{flag} = \msf{CLOSE}$
		\item $OK \leftarrow $ \{ Send $(\msf{close}, \msf{state}, \sigma_{sender}) \rightarrow \F_{\msf{contract}}$ \}
	\end{renumerate}
	\quad \Send $OK \rightarrow \mathcal{Z}$

\end{bbox}

\end{figure}

\begin{figure}[!htb]
	\newcommand{\idealqueue}{{\color{Blue} \msf{idealqueue} }}
\newcommand{\simadv}{\ensuremath{\mathcal{D}}}
\newcommand{\idealdelay}{{\color{Blue} \msf{idealdelay} }}
\newcommand{\nonce}{{\color{Blue} \msf{nonce} }}
\newcommand{\state}{{\color{Blue} \msf{state} }}
\newcommand{\simleaks}{{\color{Blue} \msf{leaks} }}
\newcommand{\flag}{{\color{Blue} \msf{flag} }}

\begin{bbox}[title={$\mathcal{S}_{pay}(P_s, P_r, b_r, b_s)$}]

Simulate real world parties $P_s', P_r'$, dummy adversary $\mathcal{D}'$, wrapper $\Wsync'$,
functionality $\F_{contract}$.

Initialize $\state = (b_s, b_r, 0)$, $\nonce = 0$, $\simleaks = \emptyset$, $\idealdelay=0$, $\idealqueue = \emptyset$
$\flag = \msf{OPEN}$

\underline{On every activation:}
	\begin{renumerate}
		\item $leaks \leftarrow \{ \Send (\msf{getleaks}) \rightarrow \mathcal{W}_{sync}\}$
		%\item $output = []$

		\item All leaks, \msf{pay} and \msf{close}, are followed by the leak of their (schedule) operation, and ideal functionality guarantees \msf{close} occurs only once and after the last \msf{pay}:
		\begin{renumerate}
			\item For each (pay, $v$) leak:
			\begin{renumerate}
				\item Update \nonce and \state to reflect the new payment 
				and saved the $rnd$,$idx$ of the corresponding (schedule) leak into 
				\idealqueue. Increment \idealdelay.
				
				\item Simulate (pay, $v$) $\rightarrow P_s'$.
			\end{renumerate}

			\item For each (close, $b_s'$, $b_r'$, $P_i$) leak:
			\begin{renumerate}
				\item Save $rnd$, $idx$ for both $P_r$ and $P_s$ in the following (schedule) 
				leaks into \idealqueue. Add $\idealdelay \pluseq 2$.

				\item \Assert $b_s = b_s'$ and $b_r = b_r'$

				\item Simulate (close) $\rightarrow P_i'$
			\end{renumerate}

			\item For all other (schedule) leaks:
			\begin{renumerate}
				\item Increment \idealdelay and save $rnd$, $idx$ into \idealqueue.
			\end{renumerate}
		\end{renumerate}

		\item Execute \msf{SimGetLeaks} 

	\end{renumerate}

\vspace{2mm} \hrule \vspace{2mm}

\OnInput \inmsg{{\em crupt} $P_i$}{msg} from $\mathcal{Z}$:
	\begin{renumerate}

		\item Simulate ({\em crupt} $P_i$', msg) $\rightarrow \mathcal{A}'$

	\end{renumerate}

\OnInput \inmsg{\msf{get-leaks}} from $\mathcal{Z}$:
	\begin{renumerate}
		\item \Send $\simleaks \rightarrow \mathcal{Z}$ 
	\end{renumerate}

\OnInput \inmsg{\msf{poll}} from $\Wsync$:
	\begin{renumerate}
		\item Execute \msf{poll}.
	\end{renumerate}

\OnInput \inmsg{\msf{delay}}{$d \token$} from $\mathcal{Z}$:
	\begin{renumerate}
		\item Simulate (\msf{delay}, $d \token$) $\rightarrow \Wsync' $
		\item $\idealdelay \pluseq d$
		\item $OK \leftarrow \{ \Send (\msf{delay}, d \token) \rightarrow \Wsync \}$
		\item $\Send OK \rightarrow \mathcal{Z}$
	\end{renumerate}

\OnInput \inmsg{\msf{exec}}{rnd}{idx} from $\mathcal{Z}$:
	\begin{renumerate}
		\item Simulate $(\msf{exec}, rnd, idx) \rightarrow \Wsync$
		\item If output $m$ from some simulated party $P_i'$:
				
			\quad Execute $\msf{SimGetLeaks}$

			\quad Execute $\msf{SimPartyOutput(m, P_i')}$

		\item Else if output $m$ from simulated $\mathcal{D}'$:
			
			\quad \Send $m \rightarrow \mathcal{Z}$

	\end{renumerate}

\end{bbox}

\end{figure}

\begin{figure}
	\begin{subfigure}{\textwidth}
		\begin{bbox}[title={Algorithm $\msf{SimGetLeaks}$}]

$leaks \leftarrow$ \{ \Send (\texttt{get-leaks}) $\rightarrow \mathcal{\mathcal{A}'}$ \}

$n \leftarrow \#$ of (schedule,...) messages in $leaks$

$idealdelay = idealdelay + n$

\{ \Send (delay, $n \token$) $\rightarrow \mathcal{W}_{sync}$ \}

\end{bbox}



	\end{subfigure}
	\newline
	\begin{subfigure}{\textwidth}
		\newcommand{\idealdelay}{{\color{Blue} \msf{idealdelay} }}

\begin{bbox}[title={Algorithm $\msf{Poll}$}]

\begin{renumerate}

  	\item $\idealdelay \minuseq 1$
  	
  	\item If $\idealdelay = 0$:
  	 
  		\quad \Send $(\texttt{delay}, 1 \token) \rightarrow \mathcal{W}_{sync}$

  		\quad $\idealdelay = 1$

  	\item \Send $(\texttt{poll},) \rightarrow \mathcal{W}_{sync}'$
 
  	\item If output $m$ from party $P_i'$:

			\quad Call $\msf{SimGetLeaks}$

			\quad Call \msf{SimPartyOutput}(m, $P_i'$)
		
		Else if $m$ from $\mathcal{A}'$:

			\quad Send $m \rightarrow \mathcal{Z}$

\end{renumerate}

\end{bbox}

	\end{subfigure}
	\newline
	\begin{subfigure}{\textwidth}
		\newcommand{\idealqueue}{{\color{Blue} \msf{idealqueue} }}
\newcommand{\idealdelay}{{\color{Blue} \msf{idealdelay} }}

\begin{bbox}[title={Algorithm $\msf{SimPartyOut}(m, P_i')$}]
	
	\begin{renumerate}
		%\item If $P_i'$ is dishonest:

		%	\quad \Send $(m, P_i') \rightarrow \mathcal{Z}$

		\item If $m$ is (pay, $v$) from $P_r'$:
		\begin{renumerate}
			\item If $P_s$ is honest:

				\quad Get $rnd$, $idx$ of this payment in \idealqueue.

				\quad Remove codeblock from \idealqueue and update queue.

				\quad \Send (exec, $rnd$, $idx$) $\rightarrow \Wsync$

			\item Else:

				\quad $OK \leftarrow$ \{ \Send (pay, $v$) $\rightarrow P_s$ \}

				\quad $leaks \leftarrow$ \{ \Send (\msf{getleaks}) $\rightarrow \Wsync$ \}

				\quad Parse $leaks$ as 2 leaks for (pay, $v$) and corresponding (schedule, $rnd$, $idx$).

				\quad \Send (\msf{exec}, $rnd$, $idx$) $\rightarrow \Wsync$
		\end{renumerate}

		\item Else if $m$ is (close) from $P_i'$:
			\begin{renumerate}
			\item If first (close) output from a simulated party:
				\begin{renumerate}
					\item If received a leak (close, $b_s'$, $b_r'$, $\cdot$) leak:

						\quad $rnd,idx \leftarrow$ deliver (close) to $P_i$

						\quad \Send (exec, $rnd$, $idx$) $\rightarrow \Wsync$

					\item Else (implies a corrupt party):
				
						\quad $OK \leftarrow$ \{ \Send (close) $\rightarrow$ corrupt $P_k'$

						\quad $leaks \leftarrow$ \{ \Send (\msf{getleaks}) $\rightarrow \Wsync$ \}

						\quad $\idealdelay \pluseq$ \# of (schedule,) leaks

						\quad Save $rnd_s,idx_s$, and $rnd_r,idx_r$ for each party.

						\quad \Send (exec, $rnd_i$, $idx_i$) $\rightarrow \Wsync$
						
				\end{renumerate}
			\item Else:

				\quad $rnd,idx \leftarrow$ codeblock for delivering (close) to $P_i$

				\quad \Send (exec, $rnd$, $idx$) $\rightarrow \Wsync$

			\end{renumerate}
	\end{renumerate}

\end{bbox}

	\end{subfigure}
\end{figure}
%\begin{figure}[h]
%	\begin{bbox}[title=$U_{pay}$]

$U_{pay} (\msf{state}, (\msf{input_L},\msf{input_R}), \msf{aux}_{in})$:

\quad \If $\msf{state} = \bot$: $\msf{state} := (0,\emptyset,0,\emptyset)$

\quad parse \msf{state} as $(\msf{cred_L},\msf{oldarr_L},\msf{cred_R},\msf{oldarr_R})$

\quad parse $\msf{aux}_{in}$ as $\{ \msf{deposits}_i \}_{i \in \{L,R\}}$

\quad \For $i \in \{L,R\}$:

	\qquad \If $\msf{input}_i = \bot$: $\msf{input}_i := (\emptyset,0)$

	\qquad parse $\msf{input}_i$ as $\msf{arr}_i,\msf{wd}_i$

	\qquad $\msf{pay}_i := 0, \msf{newarr}_i := \emptyset$

	\qquad \While $\msf{arr}_i \neq \emptyset$:

		\qqquad $e \leftarrow \msf{pop}(\msf{arr}_i)$

		\qqquad \If $e + \msf{pay}_i \leq \msf{deposits}_i + \msf{cred}_i$:

			\qqqquad $\msf{newarr}_{\neg i} \leftarrow e$

			\qqquad $\msf{pay}_i += e$

	\qquad \If $\msf{wd}_i > \msf{deposits}_i + \msf{cred}_i - \msf{pay}_i: \msf{wd}_i := 0$

\quad $\msf{cred_L} += \msf{pay_R} - \msf{pay_L} - \msf{wd_L}$

\quad $\msf{cred_R} += \msf{pay_L} - \msf{pay_R} - \msf{wd_R}$

\quad \If $\msf{wd_L} \neq 0$ or $\msf{wd_R} \neq 0$:

	\qquad $\msf{aux}_{out} := (\msf{wd_L},\msf{wd_R})$

\quad \Else: $\msf{aux}_{out} := \bot$

\quad $\msf{state} := (\msf{cred_L},\msf{newarr_L},\msf{cred_R},\msf{newarr_R})$

\quad \Return $(\msf{aux}_{out}, \msf{state})$

\end{bbox}

%	\caption{Update function for a payment channel. Given as a parameter to \Fstate. It defines the format of the \msf{state} and its updates.}
%\end{figure}
%
%\begin{figure}[h]
%	\begin{bbox}[title=$\Pi_{pay}$: $\msf{Contract_{pay}}$]

\Init $(\msf{P_L}, \msf{P_R})$:

\quad $\msf{deposits}_L, \msf{deposits}_R := 0$

% deposit($X)
\OnInput \inmsg{deposit}(tx) from \Partyi:

	\quad $\msf{deposits}_i += \msf{tx.value}$

	\quad $\msf{out}(\msf{deposits}_L, \msf{deposits}_R)$

% aux_out 
\OnInput \inmsg{output}{\msf{aux_{out}}}{\msf{tx}}:

	\quad parse $\msf{aux_{out}}$ as $(\msf{wd_L},\msf{wd_R})$

	\quad \For $i \in \{L,R\}$: $\msf{send}(P_i, \msf{wd_i})$

\end{bbox}

%	\caption{Contract pay}
%\end{figure}
%
%\begin{figure}[h]
%	\begin{bbox}[title=$\Pi_{pay}$]

Initialize $\msf{arr_i} = \emptyset, \msf{pay_i} = 0, \msf{wd_i} = 0, \msf{paid_i} = 0$

$\msf{Contract_{pay}}$ identifier $\mathcal{C}$ 

\Send $(\emptyset, 0) \rightarrow \F_{state}$

% New state from F_state
\OnInput $(\msf{cred_L}, \msf{new_L}, \msf{cred_R}, \msf{new_R})$ from $\Fstate$:

	\quad \For $e \in \msf{new_i}$:

		\qquad \textbf{Output} $(\msf{receive}, e)$

		\qquad $\msf{paid_i} += e$
	
	\quad \Send $(\msf{arr_i}, \msf{wd}-\msf{wdn}) \rightarrow \Fstate$

	\quad $\msf{arr_i} \leftarrow \emptyset$

	\quad $\msf{wdn_i} \leftarrow \msf{wd_i}$

% Pay($X)
\OnInput \inmsg{pay}{\$X} from \Env:

	\quad $\msf{Contract_{Pay}} \leftarrow \Gledger.\msf{contract}(\mathcal{C})$

	\quad \If $\$X \leq \msf{Contract_{Pay}}.\msf{deposits_i} + \msf{paid_i} - \msf{pay_i} - \msf{wd_i}$:

		\qquad $\msf{arr_i} \leftarrow \$X$

		\qquad $\msf{pay_i} += \$X$

% Withdraw($X)
\OnInput \inmsg{withdraw}{\$X} from \Env:

	\quad $\msf{Contract_{Pay}} \leftarrow \Gledger.\msf{contract}(\C)$

	\quad \If $\$X \leq \con{Pay}.\msf{deposits_i} + \msf{paid_i} - \msf{pay_i} - \msf{wd_i}$:

		\qquad $\sf{wd_i} += \$X$

\end{bbox}

%	\caption{Local protocol for parties to follow for a payment channel between two parties. Parties can pay, deposit into, or withdraw from the channel.}
%\end{figure}
%
%\begin{figure}[h]
%	\begin{bbox}[title={$\Fstate (U, \mathcal{C}, \mathcal{P} = \{P_1,...,P_N\}, \Delta)$}]

Initialize $\msf{aux}_{in} := [\bot]$, $\msf{ptr} := 0$, $\msf{state} := \emptyset$, $\msf{buf} := \emptyset, \msf{rnd} := 0$

% environment can ping the functionality to check for contract outputs
\OnInput \inmsg{ping} from \Partyi:

	\quad $\msf{aux}_{in} := \Gledger.\msf{coutput}(\mathcal{C})$

	\quad append $\msf{aux}_{in}$ to $\msf{buf}$

	\quad $j := |\msf{buf}| - 1$

	\quad $\msf{ptr} := \msf{max}(\msf{ptr},j)$ 

\vspace{2mm} \hrule \vspace{2mm}

Proceed in rounds starting at $\msf{rnd} := 0$:

	$v_{\msf{rnd},i} := \bot, \forall i \in \mathcal{P}$

	% input from the party for tihs round
\OnInput \inmsg{m} from \Partyi:

	\quad \If $v_{\msf{rnd},i} = \bot$:

		\qquad $v_{\msf{rnd},i} := m$

		\qquad \Leak $(i,v_{\msf{rnd},i}) \rightarrow \Adv$

% step function to check conditions and execute state update
\OnInput \inmsg{\msf{step}} from \Partyi:

	% if round can be progressed (deadline has passed or all parties have inpuit)
	\quad \If $\left( \forall v_{\msf{rnd},i} :  v_{\msf{rnd},i} \neq \bot \right) \vee \left( \exists v_{\msf{rnd},i} : v_{\msf{rnd},i} \neq \bot \wedge \Gledger.\msf{rnd} > \msf{deadline} \right)$:

		% compute state update
		\qquad $(\msf{state},o) := U(\msf{state}, \{v_{\msf{rnd},i}\}_{i\in\mathcal{P}}, \msf{aux}_{in}[\msf{ptr}])$
		
		\qquad $\msf{rnd} := \msf{rnd} + 1$, $\msf{deadline} := \Gledger.\msf{rnd} + \Delta$	

		\qquad \If $\left( \forall P_i : P_i.\msf{ishonest} \right)$:
			
			\qqquad $\forall P_i : \Buffer (\msf{state}, 1, P_i)$

		\qquad \Else: $\forall P_i : \Buffer (\msf{state}, O(\Delta), P_i)$

		\qquad \If $o \neq \bot$:

		% transfer money for contract output
		\qqquad \Send $(\msf{transfer},\mathcal{C},0,(output, o), \bot) \rightarrow \Gledger$ 
		

\end{bbox}

%	\caption{The ideal functionality \Fstate. The functionality proceeds in rounds and waits for parties to provide input. When all parties have provided input or the round deadline has passed, a state update is executed. Contract output is given to \Gledger in the form of a transaction. Parties must explicitly \msf{ping} the functionality in order to make progress. }
%\end{figure}
%
%\begin{figure}[h]
%	\begin{bbox}[title={$\F_{\msf{pay}} (P_s, P_r, \msf{balances}, \Delta)$}]

Initialize $\msf{flag} := \msf{OPEN}$

\vspace{2mm} \hrule \vspace{2mm}

\OnInput \inmsg{pay}{v} from $P_s$:
	\begin{renumerate}
			\item In $O(1)$ rounds:  
				
				\quad \Require $\msf{flag} = \msf{OPEN}$
				
				\quad If $\msf{balances}[P_s] >= v$:
			
				\qquad $\msf{balances}[P_s] -= v$
				
				\qquad $\msf{balances}[P_r] += v$
				
				\qquad Send $(\msf{pay}, v) \rightarrow P_r$

		\item Leak $(\msf{pay}, v)$

	\end{renumerate}

\OnInput \inmsg{close} from $P_i$:
	\begin{renumerate}	
		\item If $P_i = P_r$ or ($P_i = P_s$ and $P_s$ is honest):
			\begin{renumerate}
			\item In $O(\Delta)$ rounds:  

			\qquad \Require $\msf{flag} = \msf{OPEN}$

			\qquad In $O(1)$ rounds:  Send $(\msf{close}, b_r, b_s) \rightarrow P_s$

			\qquad In $O(1)$ rounds: Send $(\msf{close}, b_r, b_s) \rightarrow P_r$

			\end{renumerate}

		Else:
		\begin{renumerate}
			\item In $O(k \times \Delta)$ rounds:

			\qquad \Require $\msf{flag} = \msf{OPEN}$

			\qquad In $O(1)$ rounds:  Send $(\msf{close}, b_r, b_s) \rightarrow P_s$

			\qquad In $O(1)$ rounds:  Send $(\msf{close}, b_r, b_s) \rightarrow P_r$

		\end{renumerate}
		
		\item Leak $(\msf{close}, b_r, b_s, P_i)$
		
	\end{renumerate}

\end{bbox}

%	\caption{The payment channel functionality. Unlike $\Fstate$, doesn't need any notion of rounds until it must deal with on-chain transactions for deposits. Buffering for $O(\Delta)$ rounds implies the adversary can choose the number.}
%\end{figure}

