%In this section we introduce our models for asynchronous and synchronous communcation.
%Both are implemented through wrappers, $\mathcal{W}_{\msf{sync}}$ and $\mathcal{W}_{\msf{async}}$, which use the new {\em import mechanism} introduced in ~\cite{uc} and described in detail in Section~\ref{sec:prelim}.
%Before defining our wrappers we first define how ITMs leak information to the adversary.
%
%\subsection{Leakage Wrapper}
%In the original UC framework, leakage of information to the adversary means writing a message to the backdoor tape of $\mathcal{A}$.
%Leaking like this immediately passes control to the adversary an the leaking ITM can not take any further action. 
%As we describe below, ITMs often leak information {\em and} asynchronously execute code in the same activation, and it is not possible to do so when the adversary gains control.
%Instead, we use a wrapper called the {\em leakage wrapper} which stores all messages leaked from all parties and all functionalities.
%The wrapper stores the leaked msg and the import sent with it, and it returns control back to the calling ITM.
%Figure \ref{fig:wrapper:leak} describes how the wrapper works.
%For the sake of simplicity, we combine the leakage code with the communication wrappers to create one ITM that performs all the requisite tasks.
%\begin{figure}[h]
%	\begin{bbox}[title={\textbf{Wrapper} $\mathcal{W}_{\msf{leak}}$} ]

Initiaize $\msf{leaks} := \emptyset$

\vspace{2mm} \hrule \vspace{2mm}

\OnInput \inmsg{leak, msg} $d \token$ from $\F_i$/$P_i$:

	\begin{ritemize}
		\item Append $(msg, d \token)$ to \msf{leakbuffer}
		\item Send \msf{(OK)} back to the caller $\F_i$/$P_i$
	\end{ritemize}

\OnInput \inmsg{getleaks} from $\mathcal{A}$:

	\begin{ritemize}
		\item Pop all $(msg_i, d_i \token)$ and send $\{msg_1,...,msg_n\}, \sum_{i=0}^{n} d_i \token$ to $\mathcal{A}$
	\end{ritemize}

\end{bbox}

%	\caption{The leakage wrapper buffers all leaked messages and their import until the adversary requests them with \texttt{getleaks}. \mc{A} is sent all the leaks and import when requesting.}
%	\label{fig:wrapper:leak}
%\end{figure}
%
%For the remainder of the paper we make a distinction between {\em sending} a message to \mc{A} and {\em leaking} information to \mc{A}.
%For example, when \mc{Z} sends an \texttt{advance} message to the wrapper, the wrapper directly activates \mc{A} with an \texttt{advance} message.
%As a consequence of buffering all leaked messages, the adversary must always attmpted to retrieve leaks from the wrapper at every activation.
%In the real world with a dummy adversary the environment can request leaks as needed, however simulators in the ideal world must make sure to always request new leaks on every activation.
%
%\subsection{Asynchronous Wrapper}
%Our asynchonrous communication model consists of a wrapper that exists in both the real and ideal worlds.
%Unlike the works of Katz et al.~\cite{katz-clock} and Coretti et al.~\cite{coretti}, our model enables the asynchronous execution of entire codeblocks rather than just message delivery.
%Both of the previous works rely on the adversary using a unary message format for delaying the delivery of messages.
%Although this is sufficient to achieve {\em eventual delivery}\footnote{Unary format enforces a polynomial amount of delay after which the messages can not be delayed further.}, it forces protocols and funtionalities to indepdendently deal with adversarial delay and requires activations in order for messages to be {\em fetched} as in \cite{coretti}.
%In our formulation, an ITM can simply mark a codeblock to be executed asynchronously, and the wrapper takes care of enforcing its delivery with the use of the new {\em import} mechanism presented in \cite{uc}.
%In doing so, the wrapper abstraction we present here greatly removes communication model-specific artifacts from the code of functionalities and protocols. 
%Furthermore, as we will see in both the asynchronous and synchronous wrappers, the abstraction enables functionalities that are uniform accross the two models.
%The program for both functionality and protocol in the example that we present, an atomic broadcast protocol, do not depend on the communication model at play\footnote{The only difference between the two is that the synchronous wrapper accepts an additional parameter, $\Delta$, for the upper bound on round delay.}.
%
%\begin{figure}
%\centering
%	\begin{bbox}[title={\textbf{Wrapper} $\mathcal{W}_{\msf{async}}$} ]

Initialize $\msf{leakbuffer} := \emptyset, \msf{runqueue} := \emptyset, delay := 0$

\vspace{2mm} \hrule \vspace{2mm}

\OnInput \inmsg{schedule}{codeblock $e$} $1 \token$ from $\mathcal{F}_i$/$P_i$:
	\begin{ritemize}

		\item Append $e$ to $\msf{runqueue}$ with index $idx$, and increment $delay \pluseq 1$
		\item Append (schedule, $idx$) to \msf{leakbuffer} and respond to $\F_i,P_i$ with \msf{(scheduled)}

	\end{ritemize}

\OnInput \inmsg{delay} $d \token$ from $\mathcal{A}$:
	\begin{ritemize}
		\item $delay = delay + d$
	\end{ritemize}

\OnInput \inmsg{execute}{$idx$} from $\mathcal{A}$:
	\begin{ritemize}	
		\item Pop a codeblock off $\msf{runqueue}$  and execute it.
	\end{ritemize} 

\OnInput \inmsg{leak}{msg} from $\mathcal{F}_i/P_i$:
	\begin{ritemize}
		\item Append $(msg, \mathcal{F}_i/P_i, d \token)$ to $\msf{leakbuffer}$
	\end{ritemize}

\OnInput \inmsg{advance} $1 \token$ from $\mathcal{Z}$:
	\begin{ritemize}
		\item \If $delay > 0$:
			Decrement $delay \minuseq 1$ and leak \msf{(poll)} to $\mathcal{A}$
		\item \Else: 
			Pope a codeblok off $\msf{runqueue}$ and execute it.
	\end{ritemize}

\OnInput \inmsg{getleaks} from $\mathcal{A}$:
	\begin{ritemize}
		\item Sum the total amount of import in $\msf{leakbuffer}$, call it $d$.
		\item Leak $(\msf{leakbuffer},d \token)$ to $\mathcal{A}$ and empty $\msf{leakbuffer}$
	\end{ritemize}
\end{bbox}

%	\caption{The asynchronous wrapper. It accepts codeblocks from functionalities and parties, and it allows the adversary to decide execution with finite delay. The environment is also able to force progress in the protocol.}
%	\label{fig:wrapper:async}
%\end{figure}
%
%The asynchronous wrapper in Figure~\ref{fig:wrapper:async}, maintains a queue of codeblocks and a delay parameter representing the amount of delay that the adversary has assigned before the {\em next} codeblock is executed. 
%When a codeblock is added to the wrapper, the delay is incremented and information is leaked\footnote{Recall, to ``leak'' in this context means to append the message to the buffer of leaks in $\mc{W}_{\msf{leak}}$ to the adversary.} to \mc{A} about the scheduling ITM and the the codeblock's location in the queue.
%The adversary can add delay to the wrapper whenver it wants to, and can execute any codeblock in the \msf{runqueue} whenever it wants.
%Adding to the delay, however, requires the adverarsy to send 1 unit if import to the wrapper\footnote{Recall that the import mechanism was created in order to enforce polynomial run time in the whole system of ITMs spawned in the execution of a UC experiment.}.
%\todo{resolve import-delay issue brought up by shreyas}.
%
%\paragraph{Protocols in the $\{\mathcal{W}_{\msf{async}}, \F_{\msf{achan}}^{p_r,p_s}-hybrid\}$ world.}
%Protocols in the real world have access to one-shot asynchronous channel functionalities, \achan, parameterized by a sender and a receiver.
%\achan is shown in Figure ~\ref{fig:func:achan}
%
%
%\begin{itemize}
%\item The adversary has complete control over the order in which the codeblocks are executed. 
%It can send 1 unit of import and an \texttt{exec} message to the wrapper to pop any current codeblock off the queue and execute it. 
%\item As we're concerned with {\em eventual delivery}, it is insufficient to only allow the adversary to execute codeblocks and determine their delivery. 
%Therefore the environment can also force the wrapper to make progress by spending 1 unit of import to \texttt{advance} the wrapper.
%Over multiple such calls, the environment forces the delay to be 0, at which point the {\em next} codeblock in the \msf{runqueue} is removed and executed.
%\end{itemize}
%
%A critical point to address is {\em how} the environment can force the delay to reach 0.
%Even though the adversary can provide delay to the wrapper, it only has a limited amount of import that it can spend to do so~\footnote{Recall from Section \ref{sec:prelim} that a {\em balanced environment} must provide the the adversary with at least as much import as it gives to each of the other ITMs. Therefore, a simulator has finite import but enough to simulated a sandboxed real-world execution.}.
%Therefore, the adversary will eventually run out of import and lose the ability to delay the wrapper furter.
%Therefore, the bound on the delivery of the messages is unknown but upper-bounded by the polynomial runtime guaranteed by a finite amount of import provided to the adversary.
%
%
%%The model presented in this paper for both asynchronous and synchronous communication relies on the new \emph{import} mechanism described in \ref{uc} and the wrappers we present below.
%%The first departure that our wrapper make from tradition UC communication models is that they execute arbitrary code blocks rather than just deliver messages.
%%For example, a complex application may require some loging to execute some sort of state update function at a future time dependent on the communication model.
%%In the synchronous world, this may mean executing a codeblock within some number of rounds.
%%Similarly, the asynchronous world allows {\em eventual} execution of codeblocks.
%%In this section, however, we focus only on the asynchronous wrapper as its functionality is nearly identical to that of the synchronous wrapper minus some logic to ensure {\em guaranteed termination} and {\em input completeness}.
%
%The asynchronous wrapper provides an interface to functionalities and protocol parties to execute codeblocks asynchronously.
%The wrapper is shown in Figure~\ref{fig:wrapper:async}.

%\begin{itemize}
%\item describe the wrapper and how the import mechanism is used to ensure eventual delivery.
%\item Designing protocols with the wrapper
%\end{itemize}

Look at Figures \ref{fig:wrapper_async_new}, \ref{fig:dummy_adversary}, \ref{fig:dummy_party}, \ref{fig:f_async_explicit}, \ref{fig:f_rbc}, \ref{fig:prot_bracha_async}, \ref{fig:sim_bracha_async}.

\begin{figure}[!hbt]
\centering
	\begin{bbox}[title={\textbf{Wrapper} $\mathcal{W}_{\msf{async}}$} ]

Initialize $\msf{leakbuffer} := \emptyset, \msf{runqueue} := \emptyset, \msf{delay} := 0$

\vspace{2mm} \hrule \vspace{2mm}

\OnInput (\textsc{Leak}, \textsf{msg}) from $\mathcal{F}$:

	\quad Append ($\mathcal{F}$, (\textsc{Leak}, \textsf{msg})) to \textsf{leakbuffer}
	
	\quad Send \textsc{OK} to $\mathcal{F}$


\OnInput (\textsc{Eventually}, \textsf{func\_ptr}, \textsf{args}, \textsf{leak\_msg}) from $\mathcal{F}$:

	\quad Append ($\mathcal{F}$, \textsf{func\_ptr}, \textsf{args}) to \textsf{runqueue}
	
	\quad Append ($\mathcal{F}$, (\textsc{Eventually}, \textsf{leak\_msg})) to \textsf{leakbuffer}
	
	\quad Send \textsc{OK} to $\mathcal{F}$

\OnInput (\textsc{EventuallySend}, \textsf{to}, \textsf{msg}, \textsf{leak\_msg}) from $\mathcal{F}$:
	
	\quad Append (\textsc{send}, $\mathcal{F}$, \textsf{to}, \textsf{msg}) to \textsf{runqueue}

	\quad Append (\textsc{send}, (\textsc{EventuallySend}, \textsf{leak\_msg})) to \textsf{leakbuffer}

	\quad Send \textsc{OK} to $\mathcal{F}$
	
\OnInput (\textsc{Execute}, \msf{idx}) from $\mathcal{A}$:

	\quad $e \leftarrow$ pop \textsf{runqueue[idx]}

	\quad Match $e$:

		\qquad ($\mathcal{F}$, \textsf{func\_ptr}, \textsf{args}):
	
		\qqquad Send (\textsf{func\_ptr}, \textsf{args}) to $\mathcal{F}$
	
		\qquad (\textsc{send}, $\mathcal{F}$, \textsf{to}, \textsf{msg}):
			
		\qqquad Send ($\mathcal{F}$, \textsf{msg}) to \textsf{to}
	
\OnInput \textsc{GetLeaks} from $\mathcal{A}$:

	\quad \textsf{leaks} $\leftarrow$ pop all of \textsf{leakbuffer}
	
	\quad Send \textsf{leaks} to $\mathcal{A}$
	
\OnInput (\textsc{Delay}, $D \token$) from $\mathcal{A}$:

	\quad $\msf{delay} \leftarrow \msf{delay} + D$
	
	\quad Send \textsc{OK} to $\mathcal{A}$
	
\OnInput (\textsc{Advance}, $1 \token$) from \Env:

	\quad \textsf{delay} $\leftarrow$ \msf{max[delay-1, 0]}
	
	\quad If $\msf{delay} == 0$ and $\msf{runqueue}$ is non-empty:
	
		%\qquad ($\mathcal{F}$, \textsf{func\_ptr}, \textsf{args}) $\leftarrow$ pop $\msf{runqueue}[0]$
		
		%\qquad Send (\textsf{func\_ptr}, \textsf{args}) to $\mathcal{F}$

		\qquad exec subroutine for message (\textsc{Execute}, 0)
	
	\quad Else:
	
		\qquad Send \textsc{Advance} to $\mathcal{A}$
		
	
\end{bbox}

	\caption{Pseudo-code for the UC asynchronous wrapper, which is a combination of the eventually and leakage wrappers.}
	\label{fig:wrapper_async_new}
\end{figure}

\begin{figure}[!hbt]
\centering
	\begin{bbox}[title={$\mathcal{D}_\mathsf{adv}$}]

\OnInput (ITM $M$, \textsf{msg}) from \Env:

	\quad Send \textsf{msg} to $M$
	
\OnInput \textsf{msg} from ITM $M'$:

	\quad Send ($M'$, \textsf{msg}) to \Env
\end{bbox}
	\caption{Pseudo-code for the dummy adversary. The dummy adversary in the UC framework is the hardest adversary to simulate against.}
	\label{fig:dummy_adversary}
\end{figure}

\begin{figure}[!hbt]
\centering
	\begin{bbox}[title={$\mathcal{D}_\mathsf{party}$}]

If $\mathcal{D}_\mathsf{party}$ is honest:

	\quad \OnInput $x$ from \Env:
	
		\qquad Send $x$ to $\mathcal{F}$
		
	\quad \OnInput $x$ from $\mathcal{F}$:
	
		\qquad Send $x$ to \Env
	
If $\mathcal{D}_\mathsf{party}$ is corrupt:

	\quad \OnInput (ITM $M$, \textsf{msg}) from $\mathcal{A}$:

		\qquad Send \textsf{msg} to $M$
		
	\quad \OnInput \textsf{msg} from ITM $M$:
	
		\qquad Send ($M$, \textsf{msg}) to $\mathcal{A}$
\end{bbox}
	\caption{Pseudo-code for dummy parties. In the UC framework, parties are dummy parties in the ideal world.}
	\label{fig:dummy_party}
\end{figure}

\begin{figure}[!hbt]
\centering
	\begin{bbox}[title={$\mathcal{F}_\mathsf{async}^{P_i, P_j}$}]

\OnInput $x$ from $P_i$:

	\quad Send (\textsc{Eventually}, \textsf{send}, $x$, $|x|$) to $\mathcal{W}$
	
\OnInput (\textsf{send}, $x$) from $\mathcal{W}$:

	\quad Run \textsf{send}($x$)
	
Function \textsf{send}($x$):

	\quad Send $x$ to $P_j$
\end{bbox}
	\caption{Pseudo-code for the asynchronous network ideal functionality.}
	\label{fig:f_async_explicit}
\end{figure}

\begin{figure}[!hbt]
\centering
	\begin{bbox}[title={$\mathcal{F}_\mathsf{RBC}$}]

\OnInput \msf{tx} from dealer $D$:

	\quad Send (\textsc{Leak}, (\textsc{BCast}, $D$, \textsf{tx})) to $\mathcal{W}$

	\quad For each $P_i$:
	
		\qquad Send (\textsc{Eventually}, \textsf{send}, (\msf{tx}, $P_i$), (\msf{tx}, $P_i$)) to $\mathcal{W}$
	
\OnInput (\textsf{send}, (\msf{tx}, $P_i$)) from $\mathcal{W}$:

	\quad Run \textsf{send}(\msf{tx}, $P_i$)
	
Function \textsf{send}(\msf{tx}, $P_i$):

	\quad Send \msf{tx} to $P_i$
\end{bbox}
	\caption{Pseudo-code for the asynchronous reliable broadcast ideal functionality.}
	\label{fig:f_rbc}
\end{figure}

\begin{figure}[!hbt]
\centering
	\begin{bbox}[title={$\Pi_\text{Bracha}$}]

Participants: Parties $\{P_1, \dots P_n\}$, of which one assumes the role of dealer $D$

\msf{define} $\mathsf{BQ} \leftarrow \ceil{\frac{n+t}{2}} \leftarrow \text{a byzantine quorum}$

As dealer $D$:

    \quad \OnInput \textsf{tx} from \Env:
    
        \qquad For each $P_i$: send (\textsc{Val}, \msf{tx}) to $\mathcal{F}_\mathsf{Async}^{D, P_i}$
    
As party $P_i$ (including the dealer):

    \quad \OnInput (\textsc{Val}, \msf{tx}) from $D$ or (\textsc{Echo}, \msf{tx}) from \msf{BQ} parties 
    
    \quad or (\textsc{Ready}, \msf{tx}) from $\mathsf{BQ}-t$ parties:
    
        \qquad For each $P_j$: send (\textsc{Echo}, \msf{tx}) to $\mathcal{F}_\mathsf{Async}^{P_i, P_j}$
        
    \quad \OnInput (\textsc{Echo}, \msf{tx}) from \msf{BQ} parties or (\textsc{Ready}, \msf{tx}) from $\mathsf{BQ}-t$ parties:
        
        \qquad For each $P_j$: send (\textsc{Ready}, \msf{tx}) to $\mathcal{F}_\mathsf{Async}^{P_i, P_j}$
        
    \quad \OnInput (\textsc{Ready}, \msf{tx}) from \msf{BQ} parties:
    
        \qquad Send \msf{tx} to \Env
        
\end{bbox}
	\caption{Pseudo-code for the asynchronous Bracha broadcast protocol.}
	\label{fig:prot_bracha_async}
\end{figure}

\begin{figure}[!hbt]
\centering
	\begin{bbox}[title={$\mathcal{S}_\mathsf{Bracha}$}]


\textbf{Create internal emulations} of parties $\bf{P_1'},\dots,\bf{P_n'}$ running protocol $\Pi'_{Bracha}$, $n(n-1)$ point-to-point $\mathcal{F}_\mathsf{Async}$ channels, dummy adversary $\mathcal{D}_{adv}'$, and wrapper $\mathcal{W}'$

\textbf{Init} $\msf{runqueue} \leftarrow \emptyset, \msf{delay} \leftarrow 0, \msf{tx} \leftarrow \bot$

\vspace{2mm} \hrule \vspace{2mm}

On each activation, if $\msf{tx} == \bot$:

    \quad Send \textsc{GetLeaks} to $\mathcal{W}$ and receive $\msf{leaks}$
    
    \quad If $\msf{leaks} \neq \bot$:
    
        \qquad $(\textsc{Leak}, (\textsc{BCast}, D, \msf{tx})) \leftarrow \msf{leaks}[0]$
        
        \qquad Append $\msf{leaks}[1:n]$ to \msf{runqueue} and set $\msf{delay} \leftarrow \msf{delay} + n$
        
        \qquad Input \textsf{tx} to the simulated $P_i'$ corresponding to dealer $D$ 
        
        \qquad and simulate all state transitions
        
\OnInput (ITM $M$, \msf{msg}) from \Env:

    \quad If $M$ is a corrupt party $P_i$:
    
        \qquad Send ($P_i'$, \msf{msg}) to simulated $D_\msf{adv}'$ and simulate all state transitions
        
    \quad If $M$ is wrapper $\mathcal{W}$:
    
    	\qquad Send \msf{msg} to simulated wrapper $\mathcal{W}'$, and simulate all state transitions 

\OnInput \textsc{Advance} from $\mathcal{W}$:

    \quad $\msf{delay} \leftarrow \msf{max}(\msf{delay} - 1, 0)$
    
    \quad If $\msf{delay} = 0$ and $\msf{runqueue}$ is not empty:
    
        \qquad Send $(\textsc{Delay}, 1 \token)$ to $\mathcal{W}$ and receive \textsc{OK} from $\mathcal{W}$
        
        \qquad $\msf{delay} \leftarrow 1$
        
    \quad Send (\textsc{Advance}, $1 \token$) to simulated wrapper $\mathcal{W}'$, and simulate all state transitions 
    
\OnInput \msf{msg} from simulated $\mathcal{D}_\msf{adv}'$:

    \quad Send \msf{msg} to \Env
    
\OnInput \msf{tx'} from $P_i'$:

    \quad If $\msf{tx} \neq \bot$:
    
        \qquad $\msf{tx} \leftarrow \msf {tx'}$
        
        \qquad $P_i' \leftarrow$ any simulated corrupt party
        
        \qquad Send ($\mathcal{F}_\msf{RBC}, \msf{tx}$) to $P_i'$
        
        \qquad Send \textsc{GetLeaks} to wrapper $\mathcal{W}$ and receive $\msf{leaks}$
        
        \qquad Append $\msf{leaks}[1:n]$ to \msf{runqueue}
        
    \quad Pop \msf{runqueue[idx]} containing leak message (\msf{tx}, $P_i$)
    
    \quad Send (\textsc{Execute}, \msf{idx}) to $\mathcal{W}$
\end{bbox}
	\caption{Pseudo-code for the simulator for the asynchronous Bracha broadcast UC experiment.}
	\label{fig:sim_bracha_async}
\end{figure}

\clearpage

